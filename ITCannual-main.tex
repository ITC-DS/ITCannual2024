\documentclass[11pt]{jarticle}
\usepackage{ITCannual}
\usepackage{amsmath}
\usepackage{amssymb}
\usepackage{times}
\usepackage[dvipdfmx]{graphicx}

\usepackage{url}
%\usepackage[style=numeric]{biblatex}

\title{データ科学研究部門 研究報告}
\author{小林博樹, 鈴村豊太郎, 宮本大輔, 空閑洋平,\\
\textbf{河村光晶, 川瀬純也, 華井雅俊, Li Zihui}}

\begin{document}
\maketitle

\section{データ科学研究部門 概要}
データ科学研究部門では、2024年度、教授3名(特任教授1)、准教授1名、講師1名(特任講師1名)、助教3名(特任助教2名)が在籍した。同部門のメンバーは専任教員と特任教員の2つのグループから成る。専任教員はそれぞれが独立して研究活動を行うグループで、特任教員はプロジェクト等に従事する教員である。

% \subsection{専任教員グループの研究テーマ}
%
% \begin{quote}
% \begin{itemize}
% \item 計算機を介した人と生態系のインタラクションの研究(小林)
% \item 大規模グラフニューラルネットワークと様々な実社会問題への応用(鈴村)
% \item データセンタハードウェアへのソフトウェア脆弱試験の適応(空閑)
% \item Title (Kumar)
% \item Title (河村)
% \item データ駆動型知能に基づくアーバンコンピューティング(姜)
% \item 野生動物ワイヤレスセンサネットワーク実証実験基盤構築に向けた研究(川瀬)
% \item グラフニューラルネットワークとその物性予測問題への応用に関する研究(華井)
% \item Title (Zihui)
% \end{itemize}
% \end{quote}

\section{データ科学研究部門 教員研究活動}

\subsection{計算機を介した人と生態系のインタラクションの研究(小林 博樹)}

本研究室は計算機を介した人と生態系のインタラクションの研究の行っている。これまで人間を対象とした知能情報学の見地を、多様で複雑な実世界の生物・環境・地理学・獣医学領域へ応用・発展させる研究である。研究内容はコンピュータ科学、環境学、メディアアート、など多岐に亘っており、特に、計算機を介した人と生態系のインタラクションHCBI(Human-Computer-Biosphere Interaction)の概念を情報学分野で発表し、このテーマを中心に、環境問題の解決を目的として、国内外で研究活動を独自に行ってきた。古典的なコンピュータ科学では、HCI(Human-Computer Interaction)が主要な研究領域の1つとなるが、本研究室はこの研究領域を地球環境にまで拡大すべく、人間と生態系の調和あるインタラクションを実現するシステムを提案し「時空間スケールの大きい環境問題を自律的に解決する情報基盤技術」として、そのフィールドでの実証実験を試みている。つまり、コンピュータ科学の分野では人間が活動する地理空間を対象とした研究が中心であったが、本研究室は人間が活動していない、情報通信技術の応用が困難な地理空間を対象にした情報デザインと野生動物IoTの研究を行っている。このように本研究室は、情報工学をベースとして、特に計算機を用いて生態系と人間のインタラクションを専門として実績をあげている。2022年度から科学技術振興機構の創発的研究支援事業として業務を実施している。
 
% \begin{雑誌論文}{1}
 \bibitem{kobayashi1-1}
Zekun Cai, Renhe Jiang, Xinlei Lian, Chuang Yang, Zhaonan Wang, Zipei Fan, Kota Tsubouchi, Hill Hiroki Kobayashi, Xuan Song, Ryosuke Shibasaki,  "Forecasting Citywide Crowd Transition Process via Convolutional Recurrent Neural Networks", IEEE Transactions on Mobile Computing 23(5) 5433 - 5445.


 \end{雑誌論文}

 \begin{査読付}{1}
 \bibitem{kobayashi2-1}
Zhuoneng Sui, Haoran Hong, Daisuké Shimotoku, Hill Hiroki Kobayashi, "catAction: Deep learning for enhancing emotional cat-human interactions through the posture-based determination of the degrees of kittens' defensive and offensive aggressions.", Proceedings of the International Conference on Animal-Computer Interaction(ACI) 5-9, 202.  


 \end{査読付}



% %半ページから1ページが文量

\subsection{推薦システムと脳波基盤モデルの研究(鈴村 豊太郎)}

%%%% 以下、2024年度

本節では2024年度の鈴村豊太郎の研究活動について報告する。鈴村は、ニュース、Eコマースおよびオンラインサービスにおける推薦システムの研究と、神経科学における基盤モデルの研究を進めている。

ニュース領域における推薦モデルでは、記事の新しさや多様性を維持しつつユーザの関心を的確に捉える必要がある。しかし、従来手法はクリック履歴や人気度に偏りがちで、ユーザの意図的な閲覧回避行動や記事の新しさを反映できないという技術的課題がある。そこで、ユーザの回避行動と表示頻度を同時に扱う Avoidance-Aware Recommender System(AWRS)\cite{awrs}を提案し、関連性と多様性、タイムリーさを高めた推薦を実現した。AWRSの研究成果は国際会議SIAM International Conference on Data Mining (SDM25)で発表された。また、ニュース推薦における人気バイアスに対処した。人気記事の影響をネガティブなフィードバックとしてモデル化する POPKを提案し、バイアスのないユーザー嗜好表現を学習することで、推薦精度と多様性の双方を改善した。引き続き、ニュース領域では日本経済新聞社との共同研究を進めている。

Eコマース分野では、知識グラフを活用したユーザターゲティングの研究と、組合せ最適化問題を高速で解くアニーリングマシン(AM)とGNNの併用に関する研究を行った。
多様なユーザの嗜好を適切に反映するには知識グラフによる豊富な属性情報の活用が有用であることから、知識グラフの埋め込みとGNNを組み合わせて新規ターゲティング候補を抽出するAudienceLinkNet \cite{aln}を提案した。これにより、スケーラビリティと精度を両立した最適化およびユーザへのターゲティングモデルの実用性を示した。これらの研究成果はSIGIR’24 にて発表後、実運用環境での検証を進めている。
また、商品の推薦や配信先の選定が大規模な組合せ最適化問題として定式化される一方、既存の最適化ソルバではスケールやリアルタイム性に限界があるという課題がある。本研究では、AMの部分解をGNNにフィードバックして大規模グラフ上で高品質な最適解を高速に導出する手法 \cite{annealing-gnn}を提案し、AMの精度とGNNのスケーラビリティを両立する基盤を示した。

オンラインサービスにおける推薦システムでは、ユーザ行動履歴だけでなく、テキストレビューや商品説明の意味情報を統合できれば強力なパーソナライズが実現可能である。しかし、LLMの生成力を推薦モデルに直接活かす際には、数値やテキスト、グラフなど異なるデータ構造を一貫して扱うモデリング手法が必要である。本研究では、Llama-2によるIn-Context Learningで生成したアイテムプロファイルをBERTで意味的埋め込みに変換し、GNNと統合するPrompting-Based Representation Learning(P4R)\cite{p4r}を提案した。テキストに含まれる豊富なコンテキストを継続的に活用しながらユーザとアイテムの関係性を学習する全く新しい推薦フレームワークを構築し、ROEGEN@RecSys’24 ワークショップで発表後、リアルタイム推論とモデル軽量化に向けた最適化を進めている。

脳波基盤モデルの研究として、脳活動を計測するEEG信号の基盤モデルにEEGセンサ(電極)の位置関係を捉えるGNNモデルを適用した研究を行った。EEG信号は高い時間分解能を有することから時系列基盤モデルが数多く提案されてきたが、異なる電極間の空間的関係を十分に活用できていない。本研究では、電極配置による信号伝搬の関係性をグラフ構造として表現し、EEGの基盤モデルBENDRにGNNを統合することで、時間的文脈を保ちつつ電極間の相互作用を学習可能な EEG-GraphAdapter \cite{ega} と、マスクドオートエンコーダとGNNを融合したGraph-Enhanced EEG Foundation Model \cite{gefm} を提案した。これにより、EEGデータを扱う下流タスクにおいてファインチューニングの計算コストを減らしつつ、高い予測精度を達成した。これらの研究成果はAAAI-25ワークショップでポスター発表され、\cite{gefm}は国際会議 IEEE Engineering in Medicine and Biology Society (EMBC’25)にも採択された。


%%%% 以下、2023年度

% 本節では2023年度の鈴村豊太郎の研究活動について報告する。鈴村は、グラフニューラルネットワーク(GNN)、大規模言語モデル(LLM)などの深層学習を基盤にした推薦システムの研究を進めている。

% 推薦システムは、ユーザの好みやアイテム行動パターンなどをモデル化し、どのようなアイテムを購入するかを予測する問題であり、現実世界のあらゆるサービスに推薦システムが用いられていると言っても過言ではない。本年度は、地理空間領域、ニュース領域、求人マッチング領域における推薦システム、およびLLMベースの推薦システムの研究を進めている。

%  時空間における推薦システムでは、自動車の走行軌跡データに基づき次に訪問する地点(Point-of-Interest、 POI)を予測するPOI推薦において、ユーザの空間的・時間的特徴をそれぞれグラフ構造として表現し、ユーザの行動パターンを捉えるトランスフォーマベースのモデルアーキテクチャ Mobility Graph Transformer (MobGT) を提案した。この研究成果は地理情報空間におけるトップカンファレンスである国際学会 ACM SIGSPATIAL'23 に採択され、発表を行った(\cite{mobgt})。これらはトヨタ自動車との共同研究の成果でもある。

%  ニュース領域における推薦モデルでは、全ユーザの記事閲覧行動データからなる Global News Graph と記事の内容から生成された Global Entity Graph を生成し、個別のユーザの閲覧履歴と組み合わせることで、ユーザの潜在的な記事閲覧パターンを捉える記事推薦システムモデル GLORY を提案した。この論文は国際学会 ACM RecSys 2023 に採択され、Best Full Paper Runner-Up Award 及び Best Student Paper Award を受賞した(\cite{glory})。
% また、新しい記事が常に追加されるニュースサイトにおいては、過去に掲載された記事がすぐに閲覧されなくなるため、これらの記事の鮮度がニュース記事の推薦において重要である。そこで、この記事の鮮度をその内容や人気度などかを考慮して予測し、ユーザにより適切な記事を推薦するモデルを提案する。ニュース領域では日本経済新聞社との共同研究を進めている。

%  求人マッチング領域では、ユーザの嗜好のみに基づく他の一般的な推薦システムとは異なり、求人側の意向も加味して、全体的な市場のマッチングを最大化する必要がある。そこで、マッチング数を報酬とした強化学習モデルを用いることで、end-to-end でマッチング数を最大化する手法を提案した。さらに、ユーザと求人に対する実際のマッチング数の少なさを補うために、ユーザの求人への応募、マッチングを二部グラフとして表現し、GNNモデル、グラフ拡張によって精度を向上させる手法を提案した。この研究成果は、それぞれ人工知能学会全国大会 (\cite{job-jsai})
% )及び AAAI Workshop (\cite{job-aaai}) に採択され発表を行い、現在ACMの情報抽出関連の国際会議 ACM SIGIR 2024 に投稿、査読中である。また、昨年度に引き続きエス・エム・エス社との共同研究を進めており、実サービスでの検証に向けシミュレーションによる評価だけでなく A/B テストも実施した。

%  大規模言語モデル LLM (Large Language Model) ベースの推薦システムでは、数値情報のエンコーティングに関する研究を行っている。推薦モデルでは、価格や数量など数値自体が意味を持つことが多いが、現状のLLMでは数字情報を理解することが苦手であり、推薦システムへの応用のネックとなっている。そこで、既存の推薦システム用の基盤モデルに対し、pre-training 時に加算など補助的な算術演算のタスクを行うことで、数値に関する表現を部分的に捉えることを示した。この研究成果は、AAAI Workshop にて採択、発表された (\cite{num-aaai})。

%  また、前述の地理空間推薦システムにおいて、POI の情報としてテキストに、画像などマルチモーダルな情報を活用した POI推薦フレームワークを設計し、プロトタイプを実装、性能評価を続けている。ユーザが訪問した地点のジャンル、紹介文、写真をそれぞれテキストとして表現し、LLMベースの時系列推薦モデルを用いてより高精度なPOI推薦を実現することも目指している。

% また、SIGIR 2023 \cite{sigir}、 WWW 2023 \cite{kg-kp}の論文に関しては2022年度年報に報告済みであり詳細を割愛するが、2023年度に発表を行ったのでこちらに触れておく。

% 学会活動としてはAAAI 2024の Organizing Chairs として Sponsor Chairを務めた。また、2023年度から人工知能学会における理事を務め、2024年5月開催の人工知能領域における国際シンポジウム isAI 2024 (The 16th JSAI International Symposia on AI)の委員長を務めて、学会の開催準備を進めている。



%%%% 以下、2022年度

% 本節では2022年度の鈴村豊太郎の研究活動について報告する。 鈴村は、 グラフ構造に対するニューラルネットワークを用いた表現学習 Graph Neural Network (以下、GNNと呼ぶ)の基礎研究及び応用研究に取り組んでいる。 グラフ構造は、 ノードと、 ノード同士を接続するエッジから構成されるデータ構造である。 インターネット上における社会ネットワーク、 購買行動、 サプライチェーン、 金融における決済データ、 交通ネットワーク、 蛋白質相互作用・神経活動・DNAシーケンス配列内の依存性、 物質の分子構造、 人間の骨格ネットワーク、 概念の関係性を表現した知識グラフなど、 グラフ構造として表現できる応用先は枚挙に暇がない。
% \par
% 当該研究領域における研究として、時系列・動的に変化する大規模グラフに対するGNNモデルの研究を行った。
% 動的グラフに対応するGNNモデルはすでに数多く提案されているが、いずれも短期的なデータの変化しか考慮されておらず、実世界で扱われている長期的なグラフデータでは長期的なコンテキストを捉えることができない問題が潜在的に存在していた。この問題に対して、時間幅が非常に長いグラフデータの性質も捉えることができる Spectral Waveletを提案した(AAAI 2023 \cite{aaai-deft}、 Transactions on Machine Learning Research (TLMR) \cite{feta})。また、知識グラフ上で足りない関係性を補完する手法を評価する方法として トポロジカルデータ解析(Topological Data Analysis) における Persistent Homologyの概念を用いて効率的に評価する手法を提唱した(WWW 2023 \cite{kg-kp})。
% % 当該研究領域における研究として、時系列・動的に変化する大規模グラフに対するGNNモデルの研究を行った。実世界では時間幅が非常に長いデータを扱うこともあるが既存の動的グラフへのGNNの研究ではそのような点を考慮していない。この問題に対して、学習可能な Spectral Waveletを提案し、AAAI 2023 \cite{aaai-deft}、 WWW 2023 \cite{deft}、 TLMR \cite{feta}に採択された。

% GNNに関する応用研究も進めている。金融領域においては不正検出に関するGNNモデルの検証を行い、取引ネットワークをヘテロジニアスなグラフ構造に拡張することによりモデル性能の向上を達成した(KDD'22 MLG Workshop \cite{eth-gnn})。マテリアルズ・インフォマティクスの分野においては、情報基盤センターの芝隼人先生とガラス物質の形成過程モデルに対して高精度なGNNモデルを提案し\cite{botan}、また当該分野における本質的な問題に対する手法として、インバランスなデータの問題を解消するための手法 \cite{xsig-limin}および外挿のためのモデル構築を行った\cite{xsig-takashige}。E-Commerceの領域においては知識グラフを用いた商品推薦手法を提案した (ACM SIGIR 2023 \cite{sigir})。

% また、理論モデルの実世界への検証と応用サイドから意味のある研究テーマを発掘するため、企業との共同研究とも進めている。まず、自動車の走行軌跡データから次の位置や経路を予測し、ロケーションリコメンデーションなどに応用するための手法をトヨタ自動車と探求している。走行軌跡データは緯度・経度及び時刻のシーケンスデータとなるが、それを用いると運転行動パターンを捉えることができる。
% % まずシーケンスデータからグラフ構造を構築し、そのグラフ構造からGraphormerというニューラルネットワークモデルを走行軌跡データのパターンを捉えられるようなニューラルネットワークのモデルを提案した。
% この行動パターンを捉えるために、シーケンスデータをグラフ構造として表現し、Graphormerをベースにした新たなモデルを提案し、他の既存手法よりも高い精度でパターンを予測することを確認した (ECML-PKDD \cite{stgtrans} 査読中)。
% % この新たなモデルを他の既存手法と比較し、より高い精度で走行パターンを予測する事を確認した。本研究の成果を ECML-PKDD \cite{stgtrans}に提出した。
% 来年度はモビリティにおける様々な領域に応用できるように、走行軌跡データや実世界の地図データなどから事前学習モデルを構築する予定である。また、その他に都市全体の二酸化炭素排出量を抑制するために交通流を分散するための手法をこれらの事前学習モデルと深層強化学習を用いて設計・実装する予定である。

% また、エス・エム・エス社との共同研究では、介護や医療領域における人材紹介の推薦システムに関する研究を行った。超高齢化社会に突入する中、介護や医療領域における人材不足は深刻であり、より精度の高い人材マッチングが不可欠である。この問題に対して、深層強化学習を用いた人材マッチング数の最適化手法を提案し、特定の求職者・事業者側に偏ってしまう従来の推薦・マッチング手法に対して、偏りを解消できることを確認した (人工知能学会\cite{sms} 6月発表予定)。来年度に関しては更に実データでの検証を進め、企業側での要望を取り入れ、実ビジネスが持つ制約条件を取り入れた最適化モデルを提案していく予定である。また、モデルにおいて求職者と求人側での動的な二部グラフの関係性及び知識グラフを用いてより精度高いモデルを構築していく予定である。

%  日本経済新聞社(以下、日経)との共同研究も2022年10月から開始している。日経ではニュースサイトにおける記事に対してより高度な機械学習による推薦システムを目指しており、2022年度は日経側での問題設定やデータの理解を図り、研究テーマの設定に主に取り組んだ。2023年度は推薦問題や広告配信など様々な領域に応用できるように、ユーザーの行動モデルを統一的に表現する事前学習モデルを構築するべく、自己教師付き学習 (Self-Supervised Learning)を用いた手法を設計・実装する予定である。

%  また、GNNの概要と最新研究動向に関する記事を人工知能学会誌\cite{jsai-gnn}に寄稿し、Federated Learning(連合学習)の英語書籍向けに Federated Learningを用いた金融不正検知に関する手法を執筆した\cite{fl-book}。mdxプロジェクトに関する第一弾の国際学会論文として IEEE CBDCom\cite{mdx}にて論文発表を行った。


% % How Expressive are Transformers in Spectral Domain for Graphs? \cite{feta}
% % Learnable Spectral Wavelets on Dynamic Graphs to Capture Global Interactions \cite{deft}
% % Can Persistent Homology provide an efficient alternative for Evaluation of Knowledge Graph Completion Methods? \cite{kg-kp}


% % Spatio-Temporal Meta-Graph Learning for Traffic Forecasting \cite{megacrn}

% % Ethereum Fraud Detection with Heterogeneous Graph Neural Networks \cite{eth-gnn}

% % Federated Learning for Collaborative Financial Crimes Detection \cite{fl-book}



% %  本節では2021 年度の鈴村豊太郎の研究活動について報告する。 2021年4月に本学に着任し、 グラフ構造に関するニューラルネットワークを用いた表現学習 Graph Neural Network (以下、GNNと呼ぶ)の基礎研究及びその様々な応用研究に取り組んでいる。 グラフ構造は、 ノードと、 ノード同士を接続するエッジから構成されるデータ構造である。 インターネット上における社会ネットワーク、 購買行動、 サプライチェーン、 金融における決済データ、 交通ネットワーク、 蛋白質相互作用・神経活動・DNAシーケンス配列内の依存性、 物質の分子構造、 人間の骨格ネットワーク、 概念の関係性を表現した知識グラフなど、 グラフ構造として表現できる応用先は枚挙に暇がない。
% % \par
% % 当該研究領域において、時系列・動的に変化する大規模グラフに対するGNNモデルの研究を行った。分散計算環境においてスケールするGNNモデルを提唱し、その成果は高性能計算分野におけるトップカンファレンスSC2021\cite{suzumura-sc2021}に採択された。 また、金融領域における不正検知手法として、TransformerアーキテクチャをベースにしたGNN手法を提案し、国際会議 IEEE SMDS 2021\cite{suzumura-smds21}に採択された。 また、 GNNに関する招待講演\cite{suzumura-canon2021}を行った。

% % これらの研究に続いて、推薦システムへのGNNモデルに関する研究を開始している。実データ・実問題に基づいた、社会実装を見据えた研究を進めるべく、医療・介護領域における人材推薦としてエス・エム・エス社、自動車における経路推薦としてトヨタ社と共同研究を2023年4月から本格的に開始する。また、国立研究開発法人物質・材料研究機構NIMSが主導する「マテリアル先端リサーチインフラ」プロジェクトの本学拠点の一貫で、材料情報科学 Materials Informaticsへの研究も開始している。
% %  データ科学・データ利活用のためのクラウド基盤 mdx プロジェクトにおいて、 今年度は 課金付き運用開始に向けたシステム拡張、スポットVM、データ共有機構(Platform-as-a-Service)に向けた設計を進めた。 また、 mdxに関する講演活動を国内外において行った\cite{suzumura-axies2021、suzumura-nanotec2021、 suzumura-nci2021}。 mdxの論文においては、国際会議IEEE IC2E2022(10th IEEE International Conference on Cloud Engineering) に2022年3月末に投稿した。 
% % %\cite{suzumura-mdx2022}においても論文を公開した。


%




% % 
% % bibitem を作る
% \begin{雑誌論文}{1}
% \bibitem{feta}
% Bastos, Anson, Abhishek Nadgeri, Kuldeep Singh, Hiroki Kanezashi, Toyotaro Suzumura, and Isaiah Onando Mulang, "How Expressive Are Transformers in Spectral Domain for Graphs?", Transactions on Machine Learning Research (TMLR) ISSN 2835-8856, Journal of Machine Learning Research, 2022.
% \end{雑誌論文}




\begin{査読付}{6}
\bibitem{ega}
Toyotaro Suzumura, Hiroki Kanezashi, Shotaro Akahori, "Graph Adapter for Parameter-Efficient Fine-Tuning of EEG Foundation Models", The 39th Annual AAAI Conference on Artificial Intelligence (AAAI-25), The 9th International Workshop on Health Intelligence (W3PHIAI-25), 2025.

\bibitem{gefm}
Limin Wang, Toyotaro Suzumura, Hiroki Kanezashi, "GEFM: Graph-Enhanced EEG Foundation Model", The 39th Annual AAAI Conference on Artificial Intelligence (AAAI-25), Workshop on Large Language Models and Generative AI for Health (GenAI4Health), 2025.

\bibitem{awrs}
 Igor L.R. Azevedo, Toyotaro Suzumura, Yuichiro Yasui, "A Look Into News Avoidance Through AWRS : An Avoidance-Aware Recommender System", Proceedings of the 2025 SIAM International Conference on Data Mining (SDM). Society for Industrial and Applied Mathematics, 2025.


% \bibitem{popk}
% Igor L. R. Azevedo, Toyotaro Suzumura, Yuichiro Yasui, "Popular News Always Compete for the User’s Attention! POPK: Mitigating Popularity Bias via a Temporal-Counterfactual", arXiv preprint arXiv:2407.09939., 2024.

% \bibitem{edsmf}
% Igor L. R. Azevedo, Toyotaro Suzumura, "From Votes to Volatility Predicting the Stock Market on Election Day", arXiv preprint arXiv:2412.11192., 2024.

\bibitem{annealing-gnn}
Pablo Loyola, Kento Hasegawa, Andrés Hoyos-Idrobo, Kazuo Ono, Toyotaro Suzumura, Yu Hirate, Masanao Yamaoka, "Annealing Machine-assisted Learning of Graph Neural Network for Combinatorial Optimization", In NeurIPS 2024 Workshop Machine Learning with new Compute Paradigms, 2024.

\bibitem{aln}
Md Mostafizur Rahman, Daisuke Kikuta, Yu Hirate, and Toyotaro Suzumura, "Graph-Based Audience Expansion Model for Marketing Campaigns" In Proceedings of the 47th International ACM SIGIR Conference on Research and Development in Information Retrieval (SIGIR '24). Association for Computing Machinery, New York, NY, USA, 2970–2975, 2024.

\bibitem{p4r}
Chen, Junyi, Toyotaro Suzumura. "A Prompting-Based Representation Learning Method for Recommendation with Large Language Models." The 1st Workshop on Risks, Opportunities, and Evaluation of Generative Models in Recommender Systems (ROEGEN@RECSYS'24), 2024.

\end{査読付}

% \bibitem{mobgt}
% Xiaohang Xu, Toyotaro Suzumura, Jiawei Yong, Masatoshi Hanai, Chuang Yang, Hiroki Kanezashi, Renhe Jiang, and Shintaro Fukushima. 2023. Revisiting Mobility Modeling with Graph: A Graph Transformer Model for Next Point-of-Interest Recommendation. In Proceedings of the 31st ACM International Conference on Advances in Geographic Information Systems (SIGSPATIAL '23). Association for Computing Machinery, New York, NY, USA, Article 94, 1–10.

% \bibitem{glory}
% Boming Yang, Dairui Liu, Toyotaro Suzumura, Ruihai Dong, and Irene Li. 2023. Going Beyond Local: Global Graph-Enhanced Personalized News Recommendations. In Proceedings of the 17th ACM Conference on Recommender Systems (RecSys '23). Association for Computing Machinery, New York, NY, USA, 24–34.

% \bibitem{sigir}
% Md Mostafizur Rahman, Daisuke Kikuta, Satyen Abrol, Yu Hirate, Toyotaro Suzumura, Pablo Loyola, Takuma Ebisu and Manoj Kondapaka, "Exploring 360-Degree View of Customers for Lookalike Modeling",  SIGIR'23 (The 46th International ACM
% SIGIR Conference on Research and Development in Information
% Retrieval) 

% \bibitem{kg-kp}
% Anson Bastos, Kuldeep Singh, Abhishek Nadgeri, Johannes Hoffart, Toyotaro Suzumura, Manish Singh,
% "Can Persistent Homology provide an efficient alternative for Evaluation of Knowledge Graph Completion Methods?"
% In proceedings of The Web Conference (WWW), 2023.

% \bibitem{job-aaai}
% Waki, Satoshi, Toyotaro Suzumura, and Hiroki Kanezashi. "Optimizing Matching Markets with Graph Neural Networks and Reinforcement Learning." In Workshop on Recommendation Ecosystems: Modeling, Optimization and Incentive Design, AAAI 2024.

% \bibitem{num-aaai}
% Putra, Refaldi, and Toyotaro Suzumura. "On the Role of Numerical Encoding in Foundation Model of Sequential Recommendation with Sequential Indexing." In Workshop on Recommendation Ecosystems: Modeling, Optimization and Incentive Design. 2024, AAAI 2024


% \bibitem{job-jsai}
% 脇聡志, 鈴村豊太郎, 金刺宏樹, 華井雅俊, 小林秀. "強化学習によるマッチング数を最大化するジョブ推薦システム." 人工知能学会全国大会論文集 第 37 回 (2023), 一般社団法人 人工知能学会, 2023.



% \begin{著書}{2}

% \bibitem{fl-book}
% Toyotaro Suzumura, Yi Zhou, Ryo Kawahara, Nathalie Baracaldo, Heiko Ludwig,
% "Federated Learning for Collaborative Financial Crimes Detection",
% Ludwig, H., Baracaldo, N. (eds) Federated Learning. Springer, Cham. https://doi.org/10.1007/978-3-030-96896-0\_20

% \bibitem{jsai-gnn}
% 鈴村 豊太郎, 金刺 宏樹, 華井 雅俊, グラフニューラルネットワークの広がる活用分野, 人工知能, 2023, 38 巻, 2 号, p. 139-148, 公開日 2023/03/02, Online ISSN 2435-8614, Print ISSN 2188-2266, https://doi.org/10.11517/jjsai.38.2\_139

% \end{著書}

% \begin{招待講演}{3}
% \bibitem{fugaku} 鈴村豊太郎「夢の形~未来のコンピュータ~」(パネリスト)、スーパーコンピュータ「富岳」第2回成果創出加速プログラムシンポジウム「富岳百景」、2022年12月21日
% \bibitem{jsse1}鈴村豊太郎,「データ活用社会創成プラットフォームmdxおよび 大規模グラフニューラルネットワーク」, 第34回CCSEワークショップ「原子力材料研究開発におけるDX推進の現状と将来:原子力材料研究開発の革新と新展開」, 2023年2月24日
% \bibitem{jsse2}鈴村豊太郎,「人工知能の最先端研究に迫る ~大規模グラフニューラルネットワークの世界へ」, 東京大学柏キャンパス一般公開2022、特別講演会、2022年10月22日
% \bibitem{rakuten}Toyotaro Suzumura, “How Will Data and AI Change the World?”,  Rakuten Optimism Conference, Tokyo, Japan, September 29, 2022
% \bibitem{rccs}Toyotaro Suzumura,  “Large-Scale Graph Neural Networks for Real-World Industrial Applications”, The 5th R-CCS International Symposium, Kobe Japan, February 7, 2023
% \bibitem{france}Toyotaro Suzumura,  “Large-Scale Graph Neural Networks for Real-World Industrial Applications”, International Workshop “HPC challenges for new extreme scale application” held by French Alternative Energies and Atomic Energy Commission, Paris, France, March 6, 2023
% \bibitem{barcelona}Toyotaro Suzumura,  “Large-Scale Graph Neural Networks for Real-World Industrial Applications”, Barcelona Supercomputing Center, Barcelona, Spain, March 10, 2023
% \end{招待講演}



% \begin{発表}{1}
% \bibitem{sms}
% 脇 聡志, 鈴村 豊太郎, 金刺 宏樹, 小林 秀, "強化学習によるマッチング数を最大化するジョブ推薦システム." 第37回人工知能学会全国大会 (2023),  一般社団法人 人工知能学会, (2023年6月発表予定)

% \end{発表}


%\subsection{サイバーセキュリティとデータサイエンスの融合領域に関する研究(宮本 大輔)}

サイバーセキュリティにおいて,複雑に変化するサイバー脅威の傾向を大量のデータから解析し,兆候を予測することは非常に重要である.
このため、サイバーセキュリティにとってデータサイエンスは非常に重要であり、これらの領域の密接な連携に行っている.

今年度はIoT機器向けのマルウェアの特徴に注目した分析を行い,マルウェアの挙動解析を行うシステムを開発・評価をし,この成果を論文~\cite{dmiya2}としてまとめた.また,セキュリティの脆弱性の深刻度について,脆弱性情報から予測する研究に自然言語解析技法を用いる研究を行い,この成果の発表~\cite{dmiya3}を行った.

さらに以前より継続して,財務ビッグデータの解析をテーマとした共同研究を行っている.本研究では探索的なデータ解析手法を採用しているため,インタラクティブな可視化や集約を用いてデータについての歪分布の性質を有している知見を得て解析を進め,この成果を論文~\cite{dmiya1}としてまとめた.




%\begin{雑誌論文}{3}
\bibitem{dmiya1}
Masayuki Jimichi, Yoshinori Kawasaki, Daisuke Miyamoto, Chika Saka, Shuichi Nagata,
"Double-Log Modeling of Financial Data with Skew-Symmetric Error Distributions from Viewpoints of Exploratory Data Analysis and Reproducible Research",
Symmetry, MPDI, 15(9), 19 pages, September 2023. 

\bibitem{dmiya2}
Shun Yonamine, Yuzo Taenaka, Youki Kadobayashi, Daisuke Miyamoto,
"Design and implementation of a sandbox for facilitating and automating IoT malware analysis with techniques to elicit malicious behavior: case studies of functionalities for dissecting IoT malware",
Journal of Computer Virology and Hacking Techniques, Springer, 19(2), pp.149-163, March 2023.

\end{雑誌論文}


\begin{発表}{1}

\bibitem{dmiya3}
八木 裕輝, 宮本 大輔, 大規模言語モデルを用いたソフトウェア脆弱性の深刻度の推定手法, コンピュータセキュリティシンポジウム, 2023年10月

\end{発表}


\subsection{データセンタハードウェアへのソフトウェア脆弱試験の適応(空閑 洋平)}

現在のデータセンタ環境では、AIモデル作成等を高速化するためにGPU等専用アクセラレータが広く使用されている。
専用アクセラレータを用いた計算環境は、既存のCPUを中心に構成されていたクラウド型の仮想マシンクラスタから、CPUをバイパスしてデバイス間で直接データ通信するヘテロジニアス構成に移行したことで、システム全体のブラックボックス化が進んでいる。
今後、専用アクセラレータを中心としたデータセンタ環境では、CPUをバイパスするデバイス間通信が増加することで、セキュリティ監視や脆弱性試験、管理手法、データ通信内容の可視化手法といった、クラウド型クラスタ環境で実施している運用課題が顕在化すると考えられる。
本年度は、昨年度に引き続き、PIM (Processing in Memory)型のデバイスメモリの設計開発を実施し、PoCを用いたLinux NVMeドライバの既存の脆弱性の再現手法を開発し、情報処理学会コンピュータシステム・シンポジウム2024で報告した\cite{ykuga4301xyyyy}。

その他の成果としては、NIIが主催するLLM勉強会の活動に参加し、172Bモデル作成に関する活動内容を報告した\cite{ykuga4301yyyyy}。
また、クラスタワークショップ in すずかけ台2024に参加し、mdxのRDMAネットワークに関する招待講演を実施した\cite{ykuga458xxxxx}。

%\begin{査読付}{1}
\bibitem{ykuga43404131}
Ryo Nakamura, Yohei Kuga, Multi-threaded scp: Easy and Fast File Transfer over SSH, Practice and Experience in Advanced Research Computing, 2023年7月.

\end{査読付}

\begin{雑誌論文}{1}
\bibitem{ykuga45871761}
空閑洋平, 中村遼, 遠隔会議システムの計測データを用いたネットワーク品質計測, 情報処理学会論文誌, 65, 3, pp646-655, 2024年3月.

\end{雑誌論文}

\begin{招待講演}{1}
\bibitem{ykuga45871732}
空閑洋平, データセンタハードウェアへのソフトウェア脆弱試験の適応, Society 5.0時代の安心・安全・信頼を支える基盤ソフトウェア技術の構築, 2024年3月.

\end{招待講演}

\begin{招待論文}{1}
\bibitem{ykuga45871752}
河原大輔, 空閑洋平, 黒橋禎夫, 鈴木潤, 宮尾祐介, LLM-jp: 日本語に強い大規模言語モデルの研究開発を行う組織横断プロジェクト, 自然言語処理, 31, 1, pp266-279, 2024年.

\end{招待論文}

\begin{受賞}{1}
\bibitem{ykuga43010880}
空閑洋平, 山下記念研究賞, 情報処理学会(OS研究会), 2024年3月.

\bibitem{ykuga43010877}
空閑洋平, 山下記念研究賞, 情報処理学会(IOT研究会), 2024年3月.

\bibitem{ykuga43010874}
空閑洋平, 中村遼, 藤村記念ベストプラクティス賞, 情報処理学会(IOT研究会), 2023年7月.

\bibitem{ykuga9999}
中村遼, 空閑洋平, 明石邦夫, Most Innovative for HPC Uses Award, Data Mover Challenge 2023, 2024年2月.

\end{受賞}


\subsection{第一原理計算とデータ科学・機械学習による物質科学研究(河村 光晶)}

物質を構成する電子や原子に対して、基本法則となる(相対論的)量子力学や統計物理学に基づく理論的研究と、様々な環境(温度、圧力等)やプローブ(電子線、可視光、X線、中性子線、物理量測定等)における実験的研究の両者を協調的に行う事で、既存の現象の理解や新たなる物質探索をより効果的に進めることができる。
実験との比較を行うにあたり、物性物理学の理論を多様な組成$\cdot$構造を持つ現実の物質に適用するためには計算機によるシミュレーションが不可欠となり、スパコンやmdxのような高速$\cdot$大規模な並列計算資源およびデータ格納環境が利用される。
我々はそのような大規模計算機における物質科学シミュレーションの手法やプログラムの開発、およびそれを実際の物質$\cdot$現象に適用する研究を行っている。

二硫化モリブデン($\textrm{MoS}_2$)は化学的$\cdot$機械的安定性や高い電子易動度により、電子デバイスや触媒として用いられている。特に脱硫触媒としての性能において重要となるのが大気中での表面の硫黄欠陥の安定性やその周囲の構造変化である。
大気圧中での物質表面での特定の元素の周りの状況(化学的環境)を測定することは容易ではないが、間接的にそれを知ることができる方法の一つがX線光電子分光(XPS)である。この方法では、内核電子の束縛エネルギーが元素によって大きく異なることを利用して、元素選択的な信号を得ることができる。XPSを$\textrm{MoS}_2$に適用した結果では、硫黄(S)、モリブデン(Mo)ともに表面での束縛エネルギーの低下が観測された~\cite{kawamura_mos2}。これは両元素とも周囲の電子の量が増えていることを示唆している。またいくつかの表面構造モデルでの第一原理シミュレーションによれば、S原子が欠損したモデルにおいて計算で得られた束縛エネルギーが実験地と定量的に一致しており、欠損したアニオンSから両元素に電子が供給されるという機構を提案した。

極低温で金属の電気抵抗が消失する超伝導は、量子力学的性質が巨視的なスケールで発現する特異な現象であり、新たな超伝導物質の探索や発現機構の解明のための研究が行われている。
そのようなものの一つとして、既知の超伝導体$\textrm{La}_2\textrm{IRu}_2$のルテニウム(Ru)をオスミウム(Os)で置き換えた新規物質$\textrm{La}_2\textrm{IOs}_2$が合成され、この置換により超伝導転移温度($T_c$)が4.8 Kから12 Kへと大幅に変化することが観測された~\cite{kawamura_la2ios2}。RuとOsは同じ属の元素であり、非常に似た性質を持つため単体での$T_c$はほぼ同じであるが、この化合物ではLaとの軌道混成のために電子状態が大きく異なることを第一原理計算により明らかにした。またこれらの物質では磁場に対して超伝導が頑強であることも興味深い。
第一原理計算からの$T_c$予測による新規物質探索にも注目が集まっており、そのための手法開発や、現時点での理論の精度の検証もおこなっている~\cite{kawamura_yitp}。
超伝導に限らずあらたな熱電材料・構造材料の理論的探索として、$MAX$相の網羅的構造安定性解析を行った~\cite{kawamura_max}。
$MAX$相とは、d電子の少ないスカンジウム、チタン等の遷移金属元素($M$)、d電子の多い金属元素やアルミニウム等の典型元素($A$)と炭素または窒素($X$)からなる層状化合物の総称であり、構成元素の組み合わせや組成比($M_{n+1}AX_n$)により多様な物質が提案できる。
我々は密度半関数理論に基づく第一原理構造最適化と、格子振動解析を行い未だ合成されていないいくつかの$MAX$相とその長周期構造の提案を行い、フェルミ面の形状からその長周期構造の発現機構を議論した。

このように密度汎関数理論に基づく第一原理計算による記述が効果的な系がある一方で、そのような手法では捉えきれない研究対象が存在する。
強相関電子系とよばれる物質群では、非従来型超伝導や量子スピン液体といった全く新しい物理現象が発現することがあるが、それらの機構を解明するためのシミュレーションには厳密対角化や量子モンテカルロ法といったより高度な計算手法が必要となる。
第一原理計算の結果を前処理として、単純なモデル化を経て高度な解析を行うためのぽうろグラムパッケージの開発を行っており、
厳密対角化を行う$\mathcal{H}\Phi$~\cite{kawamura_hphi,kawamura_hpci3}と多変数変分量子モンテカルロ法をおこなうmVMC~\cite{kawamura_hpci2}がある。
これらのプログラムは東京大学物性研究所のソフトウェア高度化プロジェクトのなかで開発が行われた~\cite{kawamura_hpci1, kawamura_kotai}。

%\begin{招待講演}{1}
\bibitem{kawamura_yitp}
河村光晶,
``超伝導密度汎関数理論の精度検証と応用'',
超伝導研究の発展と広がり,
京都大学基礎物理学研究所,
2023年12月.
\end{招待講演}

\begin{招待論文}{1}
\bibitem{kawamura_kotai}
吉見一慶, 本山裕一, 青山龍美, 川島直輝, 河村光晶,
物性研の計算物性科学コミュニティ支援活動,
固体物理 Vol. 58, No. 9, 29 (2023).
\end{招待論文}

\begin{受賞}{1}
\bibitem{kawamura_hpci1}
HPCIソフトウェア賞【普及部門賞】最優秀賞, 
「計算物質科学ソフトウェア普及活動」, 
MateriAppsチーム (井戸 康太、福田 将大、笠松 秀輔、三澤 貴宏) および PASUMSチーム (本山 裕一, 河村 光晶, 吉見 一慶), 2
023年5月.

\bibitem{kawamura_hpci2}
HPCIソフトウェア賞【開発部門賞】最優秀賞,
「mVMC」, 
mVMC開発チーム (井戸 康太, 森田 悟史, 吉見 一慶, 本山 裕一, 加藤 岳生, 河村 光晶, Ruquing Xu, 今田 正俊, 三澤 貴宏), 
2023年5月.

\bibitem{kawamura_hpci3}
HPCIソフトウェア賞【開発部門賞】優秀賞, 
「$\mathcal{H}\Phi$」, 
$\mathcal{H}\Phi$開発チーム (河村 光晶, 吉見 一慶, 三澤 貴宏, 井戸 康太, 本山 裕一, 山地 洋平), 
2023年5月.
\end{受賞}


\begin{雑誌論文}{4}
\bibitem{kawamura_la2ios2}
H. Ishikawa, T. Yajima, D. Nishio-Hamane, S. Imajo, K. Kindo, and M. Kawamura,
``Superconductivity at 12 K in ${\mathrm{La}}_{2}{\mathrm{IOs}}_{2}$: A $5d$ metal with osmium honeycomb layer'',
Physical Review Materials \textbf{7}, 054804 (2023).

\bibitem{kawamura_hphi}
K. Ido, M. Kawamura, Y. Motoyama, K. Yoshimi, Y. Yamaji, S. Todo, N. Kawashima, T. Misawa,
``Update of $\mathcal{H}\Phi$: Newly added functions and methods in versions 2 and 3'',
Computer Physics Communications \textbf{298}, 109093 (2024).

\bibitem{kawamura_mos2}
F. Ozaki, Dr. S. Tanaka, Y. Choi, W. Osada, K. Mukai, M. Kawamura, M. Fukuda, M. Horio, T. Koitaya, S. Yamamoto, I. Matsuda, T. Ozaki, and J. Yoshinobu,
``Hydrogen-induced Sulfur Vacancies on the $\mathrm{MoS}_2$ Basal Plane Studied by Ambient Pressure XPS and DFT Calculations'',
ChemPhysChem e202300477 (2023).

\bibitem{kawamura_max}
M. Khazaei, S. Bae, R. Khaledialidusti, A. Ranjbar, H. Komsa, S. Khazaei, M. Bagheri, V. Wang, Y. Mochizuki, M. Kawamura, G. Cuniberti, S. M. V. Allaei, K. Ohno, H. Hosono, and H. Raebiger, 
``Superlattice MAX Phases with A-Layers Reconstructed into 0D-Clusters, 1D-Chains, and 2D-Lattices'',
The Journal of Physical Chemistry C \textbf{127}, 30, 14906 (2023).

\end{雑誌論文}



\subsection{野生動物ワイヤレスセンサネットワークと時空間行動分析に関する研究(川瀬 純也)}
本研究室では、野生動物装着型ワイヤレスセンサーネットワーク機構による自然環境下でのデータ収集手法の開発と、それによって得られるデータの解析手法についての研究を行っている。人間が容易には侵入できないエリアでの継続的なデータ収集機構として種々の社会問題の解決に寄与することを目指している。

野生動物を対象に含む時空間行動分析の手法として、配列アライメント手法を用いた類型化手法に着目し、研究を進めている。野生動物の時空間行動は、その意図や目的などを明示的に把握することができず、GPSデータなどの移動データから推測するしかない。そのGPSデータも自然環境下では測位精度が低く、野生動物装着型デバイスのバッテリー持続期間の問題からも、測位の間隔やタイミングが整ったデータを収集することは困難である。多種多様で、大量かつ欠落部分を含む移動データを分析する上で、これらの問題点を考慮した定量的な類型化手法は非常に重要となる。

2024年度は、2023年度から引き続き北海道の広い放牧地で自由に移動し活動する乳牛のGPSデータ等の収集を行い、6か月に渡る継続的な乳牛のGPSデータを収集することができた。また2023年度に収集した同様の乳牛のGPSデータと合わせて、これらを用いて類型化手法の検討を進めてきた。観光地などでの人の移動行動を対象とした類型化の既存手法では、対象エリア内を機能や空間的なつながりにもとづいて分割し、類型化に用いる。しかしこの手法は、隔たりのない自然環境を自由に移動する動物の移動行動には適用できないと考えられた。そこで本研究では実際の乳牛たちの移動行動の特徴にもとづいて、類型化の単位(ここでは1日24時間)ごとに対象エリアの分割を行えるようにした。また、1日毎の類型化結果をもとに、分析期間(約4カ月間)を通した類型化を行った。これにより、乳牛たちの日々の移動行動の類似性と、分析期間を通したその類似性の変化をもとに、移動行動の類型化を行うことが可能であることが示された。以上については、動物行動関係学会と地理情報科学関係学会の2カ所で報告を行った。\cite{kawase01, kawase02}また、2024年度に収集したGPSデータも合わせて対象とし、手法の妥当性の検討を進めている。

さらに、遅延耐性ネットワーク (DTN)技術を用いたワイヤレスセンサーネットワークのシミュレーションを行い、被覆面積の効率的な最大化手法について検討している。DTN技術を用いた野生動物装着型ワイヤレスセンサーネットワークにおいては、異なる群れの間を行き来したり、積極的に他の個体と接触したりする個体の存在が重要となる。そのような個体が、広くデータを伝播させる役割を担うことができると考えられるからである。類型化手法においては、「一緒に行動する群れ」を特定するだけでなく、「群れの間を行き来するような特徴的な少数派」を効率的に見つけ出すことを目的のひとつとしている。これらに着目し、継続的に研究を行っていく。
%\begin{発表}{1}

\bibitem{kawase01}
川瀬純也: 配列アライメントを用いた放牧牛の移動行動クラスタリング手法の検討, 動物の行動と管理学会 2024年度研究発表会, 熊本, 2024.9

\bibitem{kawase02}
川瀬純也: 配列アライメントを用いた放牧牛の移動行動クラスタリングの試み, 第33回地理情報システム学会学術研究発表大会, P2-19, 京都, 2024.10

\end{発表}

\subsection{材料データの収集および解析のための情報システム基盤に関する研究(華井 雅俊)}

本節では、2024年度の華井雅俊の研究活動について報告する。
今年度は昨年度に引き続き、大規模材料科学データの収集および解析のための情報システム基盤に関する研究開発を主に取り組んだ。

近年、機械学習分野の社会的な盛り上がりやマテリアルズインフォマティクスの発展などから、材料科学データの重要性がますます強調されている。
材料の開発研究においてデータはおおよそ2つに分類され、1つは理論計算によって生み出されるシミュレーションデータ、もう1つは実際の材料実験装置 (電子顕微鏡や放射光装置) から得られる実験データである。
今日における個々の材料研究開発において、それらの相互的なデータ解析・データ同化は不可欠であり広く一般的に行われている。一方で、実験とシミュレーションを統合した大規模なデータ収集や利活用、更に汎用的に利用可能な大規模データセット構築などは未だ決定的な提案がなく、課題が多い。
機械学習やグラフニューラルネットワークネットワークなどよりハイレベルなデータ駆動型研究を支える基礎インフラとして、本研究では材料用大規模データ集積・解析基盤を構築し主にシステム面の問題解決に注力している。

具体的には、東京大学情報基盤センター、同大学大学院工学系研究科、日本原子力研究開発機構、広島大学、理化学研究所、情報・システム研究機構 統計数理研究所のメンバーで共同開発の、ARIM-mdxデータシステム~\cite{hanai-arim-mdx}の運用や利用拡大を実施した。システムの基盤技術と1年間の利用動向をまとめ、国際会議IEEE BigData 2024にて発表した~\cite{hanai-BigData,hanai-press}。
また、昨年度出願の特許をベースに、株式会社Haudiとの共同開発にてARIM-mdxで利用されるIoTデバイスの商品化を実施した~\cite{hanai-rxt}。
今年度1000ユーザーの大台を超え、材料研究向けデータシステムとして着実な発展を行うことができ、また各招待講演によって発表の機会を多く持つことができた~\cite{hanai-kyudai,hanai-akiba,hanai-simpo,hanai-mdx,hanai-nbci}。


% 本節では、2023年度の華井雅俊の研究活動について報告する。
% 今年度は主に大規模材料科学データの収集および解析のための情報システム基盤に関する研究開発を主に取り組んだ。

% 近年、機械学習分野の社会的な盛り上がりやマテリアルズインフォマティクスの発展などから、材料科学データの重要性がますます強調されている。
% 材料の開発研究においてデータはおおよそ2つに分類され、1つは理論計算によって生み出されるシミュレーションデータ、もう1つは実際の材料実験装置 (電子顕微鏡や放射光装置) から得られる実験データである。
% 今日における個々の材料研究開発において、それらの相互的なデータ解析・データ同化は不可欠であり広く一般的に行われている。一方で、実験とシミュレーションを統合した大規模なデータ収集や利活用、更に汎用的に利用可能な大規模データセット構築などは未だ決定的な提案がなく、課題が多い。
% 機械学習やグラフニューラルネットワークネットワークなどよりハイレベルなデータ駆動型研究を支える基礎インフラとして、今年度の本研究では材料用大規模データ集積・解析基盤を構築し主にシステム面の問題解決に注力した。

% 具体的には今年度にかけて東京大学工学系研究科と共同で、東京大学の共用材料実験施設の運用プロジェクト (ARIM) 向けにARIM-mdxデータシステムを開発し~\cite{hanai-arim-mdx}、一般ユーザーの利用を開始した。
% 8月末の運用開始から3月末までに学内・学外および企業ユーザーを含む237ユーザーに達し、着実なスタートを切ることができた。
% 東大の武田クリーンルームユーザーへの展開、北大・九大・産総研を含む学外展開を進めており来年度以降はさらなるユーザー獲得に注力する。

% 本データシステムは主に3つの機能からなり、1つは総量3PBの超大容量クラウドストレージ、2つめはIoTデバイスによる直接データ転送、3つめはK8sによる分散コンテナ計算環境をベースとした Jupyter, VSCode, RemoteDesktop環境である。
% 特に2つ目のIoTデバイスにおいては、これまでデータ取りだしのクラウド化が難しかった非ネットワーク計測装置に、独自開発のIoTを接続し経由することで、クラウドストレージ環境に直接データ転送を行うことができるようになった。
% 本IoT上で動作するデータ転送アルゴリズムはユーザー認証および耐故障性をもち、各ユーザーがそれぞれの貴重な実験データを安全にクラウドストレージに通信することが可能となった。
% 技術的な詳細は\cite{hanai-UCC}にまとめ、IoT/クラウドに関する国際会議にて発表を行った。
% 本IoTデバイスの技術は、東京大学TLOにその社会的なニーズを認められ、特許出願 ~\cite{hanai-iot} に結びついたため、来年度、パートナー企業との機能の充実および社会実装を積極的に進める。




% \cite{hanai-xiaohang}
% \cite{hanai-iot}
% \cite{hanai-utac}
% \cite{hanai-arim-mdx}




% 本節では、2022年度の華井雅俊の研究活動について報告する。グラフニューラルネットワーク(Graph Neural Network, GNN) とその物性予測問題への応用に関する研究に取り組んでいる。
% 電池、半導体、触媒、医薬品などの材料開発・材料研究の全般において、膨大にある候補材料のさまざまな物性を比較解析することが不可欠であるが、それら候補全てを実際に作り検証することは現実的でない。そのため分子構造などの比較的簡単に得られる物質情報から目的の物性を予測・計算することが重要である。近年では、分子構造(グラフ)データとグラフニューラルネットワークを利用した物性値予測モデルの研究が盛んになってきている。2021年度に引き続き、Stanford Universityが取りまとめるOpen Graph Benchmark (OGB) やCMUとFacebookが主導するOpen Catalyst Project (OCP) などの物性予測問題ベンチマークが機械学習系研究コミュニティで取り上げられ、ますますの盛り上がりを見せている。

% 2022年度は、GNNを用いた物理問題へのアプローチに関して大きく2の方向性から取り組んでいる。1つは、既存GNNモデルを物理の問題へ応用した際に現れる機械学習手法の限界に関する研究である。機械学習で典型的な、画像処理や自然言語処理では注力されないが物理の問題では非常に重要となる外挿予測とデータの不均衡性に関して特に取り組んだ~\cite{xsig-limin,xsig-takashige}。
% もう1つは対象の物質により注力した応用研究である。具体的には、ガラスのダイナミクスの予測問題に着手した~\cite{botan}。 ガラスの振る舞いをグラフを用いてモデル化しGNNを用いることで、分子動力学などのシミュレーション結果を詳細に予測した。
% また、その他のGNN応用とも共通の課題として鈴村研究室メンバーとの共同研究も行っており、例えば交通システムの問題に関して研究を行った~\cite{stgtrans}。

% また、業務では情報基盤センターが進めるmdxに関して、物性研究や材料開発で得られるデータの利活用を進めている。本年度は物性データに特化したペタスケールストレージをmdxに連携させるシステムを設計し導入を行った。

% % 一般に、ある物性値が広範囲な材料群に対し既知である場合予測モデルを構築することが可能となるが、しかし一方で、多くの物性値においては既知である材料が少数であり学習データが不足しているため、実用精度の予測モデルを構築することは難しい。同一の物性であってもパラメータや実験条件が共通化されていないと予測モデルの構築は難しいことが知られ、既存の物性予測の研究では、共通の条件で整理された大規模データが主に利用される(例えば、上のコンペティションなど)。小規模に限定されるデータ、例えば計算コストの膨大なシミュレーション値や実験データ、において、機械学習の利用は限定的であり、大きな研究課題の1つとなっている。

% % 我々の研究チームはこのような少規模データに着目し研究を開始した。2021年度下半期は新手法提案への準備としてデータの収集に注力し研究を行った。機械学習分野や材料研究分野で用いられるオープンデータに加え、同学の工学部の研究チームへコンタクトし、スパコンスケールの計算資源を利用し得られた高価なシミュレーション値や実際の実験データに関してヒアリングを行い、データ収集を開始した。
% % また、本部門で開発のすすめるmdxにおいては材料系研究への利用促進を行っており、本研究の中間報告として第20回ナノテクノロジー総合シンポジウムにて発表し、IEEE IC2E 2022への投稿論文にて材料系研究におけるクラウド基盤の利活用をまとめた。

% % % 2021年度は主に、分野の調査と


% % 本節では、2021年度の華井雅俊の研究活動について報告する。2021年9月の本学着任から、グラフニューラルネットワークとその物性予測問題への応用に関する研究に取り組んでいる。

% % 電池、半導体、触媒、医薬品などの材料開発・材料研究の全般において、膨大にある候補材料のさまざまな物性を比較解析することが不可欠であるが、それら候補全てを実際に作り検証することは現実的でない。そのため分子構造などの比較的簡単に得られる物質情報から目的の物性を予測・計算することが重要である。近年では、分子構造(グラフ)データとグラフニューラルネットワークを利用した物性値予測モデルの研究が盛んになってきている。特に2021年度はStanford Universityが取りまとめるOpen Graph Benchmark (OGB) やCMUとFacebookが主導するOpen Catalyst Project (OCP) などの機械学習系研究コミュニティのコンペティションで物性予測問題が取り上げられた初めての年であった。

% % 一般に、ある物性値が広範囲な材料群に対し既知である場合予測モデルを構築することが可能となるが、しかし一方で、多くの物性値においては既知である材料が少数であり学習データが不足しているため、実用精度の予測モデルを構築することは難しい。同一の物性であってもパラメータや実験条件が共通化されていないと予測モデルの構築は難しいことが知られ、既存の物性予測の研究では、共通の条件で整理された大規模データが主に利用される(例えば、上のコンペティションなど)。小規模に限定されるデータ、例えば計算コストの膨大なシミュレーション値や実験データ、において、機械学習の利用は限定的であり、大きな研究課題の1つとなっている。

% % 我々の研究チームはこのような少規模データに着目し研究を開始した。2021年度下半期は新手法提案への準備としてデータの収集に注力し研究を行った。機械学習分野や材料研究分野で用いられるオープンデータに加え、同学の工学部の研究チームへコンタクトし、スパコンスケールの計算資源を利用し得られた高価なシミュレーション値や実際の実験データに関してヒアリングを行い、データ収集を開始した。
% % また、本部門で開発のすすめるmdxにおいては材料系研究への利用促進を行っており、本研究の中間報告として第20回ナノテクノロジー総合シンポジウムにて発表し、IEEE IC2E 2022への投稿論文にて材料系研究におけるクラウド基盤の利活用をまとめた。

% % % 2021年度は主に、分野の調査と




%\begin{招待講演}{3}

\bibitem{hanai-simpo}
華井雅俊、
"ARIM-mdxデータシステム",
ARIM「第2回革新的なエネルギー変換を可能とするマテリアル領域」シンポジウム,
2024年1月

\bibitem{hanai-utac}
華井雅俊,
"ARIM-mdxデータシステム: 材料実験データの利活用に向けた実験施設のDX化", 
第一回UDAC-SRIS合同勉強会,
2023年12月

\bibitem{hanai-gizyutsu}
華井雅俊、
"ARIM-mdxデータシステム"
ARIM 第1回計測技術スタッフ全体研修会,
2023年12月

\end{招待講演}

\begin{査読付}{3}

\bibitem{hanai-UCC}
Masatoshi Hanai, Mitsuaki Kawamura, Ryo Ishikawa, Toyotaro Suzumura, Kenjiro Taura
"Cloud Data Acquisition from Shared-Use Facilities in A University-Scale Laboratory Information Management System."
In Proceedings of the 16th IEEE/ACM International Conference on Utility and Cloud Computing (UCC 2023), December 5, 2023, Taormina.

\bibitem{hanai-xiaohang}
Xiaohang Xu, Toyotaro Suzumura, Jiawei Yong, Masatoshi Hanai, Chuang Yang, Hiroki Kanezashi, Renhe Jiang, Shintaro Fukushima, 
"Revisiting Mobility Modeling with Graph: A Graph Transformer Model for Next Point-of-Interest Recommendation."
In Proceedings of the 31st ACM International Conference on Advances in Geographic Information Systems (SIGSPATIAL ’23). Association for Computing Machinery, New York, NY, USA, Article
94, 1–10.

\bibitem{job-jsai}
脇聡志, 鈴村豊太郎, 金刺宏樹, 華井雅俊, 小林秀. "強化学習によるマッチング数を最大化するジョブ推薦システム." 人工知能学会全国大会論文集 第 37 回 (2023), 一般社団法人 人工知能学会, 2023.

\end{査読付}


\begin{特許}{1}
\bibitem{hanai-iot}
華井雅俊、河村光晶、石川亮、鈴村豊太郎、
“IoT デバイス、データ転送システムおよびデータ転送方法”,
特願2023-156343,
東大TLOより出願済,
\end{特許}

\begin{公開}{1}
\bibitem{hanai-arim-mdx}
"ARIM-mdxデータシステム",
\url{https://lcnet.t.u-tokyo.ac.jp/data_system/},
\end{公開}

% \begin{雑誌論文}{1}

% \bibitem{gnn-glass}
% Hayato Shiba, Masatoshi Hanai, Toyotaro Suzumura, and Takashi Shimokawabe, "BOTAN: BOnd TArgeting Network for prediction of slow glassy dynamics by machine learning relative motion." The Journal of Chemical Physics 158, no. 8, 084503, 2022.
% \end{雑誌論文}

% \begin{査読付}{3}

% \bibitem{xsig-limin-hanai}
% Limin Wang, Masatoshi Hanai, Toyotaro Suzumura, Shun Takashige, Kenjiro Taura, "On Data Imbalance in Molecular Property Prediction with Pre-training" xSIG 2023 (submitted)

% \bibitem{xsig-takashige-hanai}
% Shun Takashige, Masatoshi Hanai, Toyotaro Suzumura, Limin Wang, Kenjiro Taura, "Is Self-Supervised Pretraining Good for Extrapolation in Molecular Property Prediction?" xSIG 2023 (submitted)

% \bibitem{stgtrans-xiaohang}
% Xiaohang Xu, Toyotaro Suzumura, Jiawei Yong, Masatoshi Hanai, Chuang Yang, Hiroki Kanezashi, Renhe Jiang, Shintaro Fukushima, "Spatial-Temporal Graph Transformer for Next Point-of-Interest Recommendation", Machine Learning and Knowledge Discovery in Databases: European Conference, (ECML-PKDD), 2023 (submitted)

% \end{査読付}


\subsection{大規模言語モデルの研究とその応用(Li Zihui)}

Educational Large Language Models (LLMs): This research area focuses on enhancing the discovery of educational resources and the interpretability of transformer models in NLP education. We explored whether traditional methods could effectively identify high-quality study materials and proposed a transfer learning-based pipeline to improve resource discovery, especially for generating introductory paragraphs in educational content \cite{ireneli-3,ireneli-9}. Additionally, we assessed the potential of LLMs as tools for learning support, presenting a benchmark to evaluate their performance in various NLP tasks \cite{ireneli-5}. Our investigation also included the generation of concise survey articles in the NLP domain using LLMs. While GPT-created surveys were found to be more up-to-date and accessible than human-written ones, some limitations, such as occasional factual inaccuracies, were noted \cite{ireneli-11}. Furthermore, we explored the use of LLMs in teaching complex legal concepts through narrative methods \cite{ireneli-8}. Our LLM research works \cite{ireneli-1} have been noticed by \textbf{Nature News}, and I had the opportunity to be interviewed by a reporter, where I shared my insights on the impact of Large Language Models (LLMs) from the academia perspective \cite{ireneli-13}.

Medical Large Language Models (LLMs): In medical LLM research, we focus on enhancing LLMs' capabilities in medical question answering and teaching complex legal concepts via storytelling. Collaborating with MatsuoLab, we developed a framework combining knowledge graphs and ranking techniques to boost LLMs' effectiveness in medical queries \cite{ireneli-7}. We also introduced Ascle, a Python NLP toolkit for medical text generation \cite{ireneli-10}.

Other Benchmarks for NLP Open Questions: Our research extends to improving knowledge graph completion, evaluating LLMs in NLP problem-solving, and developing medical text generation tools. We investigated the enhancement of knowledge graph completion using node neighborhood data \cite{ireneli-4} and explored methods to better interpret transformer models by emphasizing crucial information \cite{ireneli-6}. Besides, we have been investigating other branches including graph methods for news encoding \cite{ireneli-2,ireneli-12}.



%% \begin{雑誌論文}{1}

% \bibitem{sample}
% Irene Li, Jessica Pan, Jeremy Goldwasser, Neha Verma, Wai Pan Wong, Muhammed Yavuz Nuzumlalı, Benjamin Rosand, Yixin Li, Matthew Zhang, David Chang, R. Andrew Taylor, Harlan M. Krumholz, Dragomir Radev,\lq\lq Neural Natural Language Processing for unstructured data in electronic health records: A review", Computer Science Review, volume 46, 2022.

% \end{雑誌論文}

% \begin{査読付}{1}

% \bibitem{feng2022diffuser}
% Aosong Feng and Irene Li and Yuang Jiang andRex  Ying,\lq\lq Diffuser: Efficient Transformers with Multi-hop Attention Diffusion for Long Sequences", Proceedings of Thirty-Seventh AAAI Conference on Artificial Intelligence (AAAI), 2023.

% \end{査読付}

% \begin{発表}{1}

% \bibitem{li2023nnkgc}
% Zihui Li and Boming Yang and Toyotaro Suzumura,\lq\lq NNKGC: Improving Knowledge Graph Completion with Node Neighborhoods", arXiv preprint, 2023.

% \end{発表}

% \begin{招待講演}{1}  % invited talks

% sample
% \bibitem{sample-kobayashi3-1}
% Hill Hiroki Kobayashi, mdx: A Cloud Platform for Supporting Data Science and Cross-Disciplinary Research Collaborations, the Nepal JSPS Alumni Association (NJAA), hosted its 7th Symposium, 29 November, 2022.

% \bibitem{ireneli-1}
% Irene Li, A Journey from Transformers to Large Language Models: an Educational Perspective, 2023 the 1st International Conference on AI-generated Content (AIGC2023), Aug, 2023


% \end{招待講演}  % end: invited talks

% \begin{招待論文}{1}  % invited papers



% \end{招待論文}  % end: invited papers


% \begin{受賞}{1}  % awards

% \bibitem{ireneli-2}
% Boming Yang, Dairui Liu, Toyotaro Suzumura, Ruihai Dong and Irene Li,\lq\lq Going Beyond Local: Global Graph-Enhanced Personalized News Recommendations", Proceedings of the 17th ACM Conference on Recommender Systems  (RecSys 2023), 2023 (Best Student Paper Award)

% \end{受賞}  % end: awards


% \begin{著書}{1}  % books

% \end{著書}  % end: books


\begin{雑誌論文}{1}  % journals

% % sample
% % \bibitem{sample-kobayashi1-3}
% % Wenjing Li, Haoran Zhang, Jinyu Chen, Peiran Li, Yuhao Yao, Xiaodan Shi,  Mariko Shibasaki, Hill Hiroki Kobayashi, Xuan Song and Ryosuke Shibasaki, \lq\lq Metagraph-Based Life Pattern Clustering With Big Human Mobility Data", IEEE Transactions on Big Data, Feb, 2023.

\bibitem{ireneli-1}
Rui Yang, Qingcheng Zeng, Keen You, Yujie Qiao, Lucas Huang, Chia-Chun Hsieh, Benjamin Rosand, Jeremy Goldwasser, Amisha D. Dave, Tiarnan D. L. Keenan, Yuhe Ke, Cheng Hong, Nan Liu, Emily Y. Chew, Dragomir Radev, Zhiyong Lu, Hua Xu, Qingyu Chen, and Irene Li,\lq\lq Ascle—A Python Natural Language Processing Toolkit for Medical Text Generation: Development and Evaluation Study," Journal of Medical Internet Research, vol. 26, 2024.


\bibitem{ireneli-2}
Yuhe Ke, Rui Yang, Sui An Lie, Taylor Xin Yi Lim, Yilin Ning, Irene Li, Hairil Rizal Abdullah, Daniel Shu Wei Ting, and Nan Liu, \lq\lq Mitigating Cognitive Biases in Clinical Decision-Making Through Multi-Agent Conversations Using Large Language Models: Simulation Study,” Journal of Medical Internet Research, vol. 26, 2024.

\bibitem{ireneli-3}
Han Yuan, Mingcheng Zhu, Rui Yang, Han Liu, Irene Li, Chuan Hong, \lq\lq Rethinking Domain-Specific Pretraining by Supervised or Self-Supervised Learning for Chest Radiograph Classification: A Comparative Study Against ImageNet Counterparts in Cold-Start Active Learning," Health Care Science, vol. 3, no. 2, 2025.

\bibitem{ireneli-4}
Dairui Liu, Derek Greene, Irene Li, Xuefei Jiang, and Ruihai Dong, \lq\lq Topic-Centric Explanations for News Recommendation," ACM Transactions on Recommender Systems, vol. 3, no. 2, 2024, Article 9.

\end{雑誌論文}  % end: journals


\begin{査読付}{1}  % papers (peer-reviewed)

\bibitem{ireneli-5}
Dairui Liu, Boming Yang, Honghui Du, Derek Greene, Neil Hurley, Aonghus Lawlor, Ruihai Dong, and Irene Li, \lq \lq RecPrompt: A Self-Tuning Prompting Framework for News Recommendation Using Large Language Models," Proceedings of the 33rd ACM International Conference on Information and Knowledge Management (CIKM), 2024.

\bibitem{ireneli-6}
Yingjian Chen, Haoran Liu, Yinhong Liu, Jinxiang Xie, Rui Yang, Han Yuan, Yanran Fu, Pengyuan Zhou, Qingyu Chen, James Caverlee, and Irene Li, \lq\lq GraphCheck: Breaking Long-Term Text Barriers with Extracted Knowledge Graph-Powered Fact-Checking," Proceedings of the 63rd Annual Meeting of the Association for Computational Linguistics (ACL), 2025.

% sample
% \bibitem{sample-kobayashi2-1}
% Daisuk\'e Shimotoku, Tian Yuan, Laxmi Kumar Parajuli and Hill Hiroki Kobayashi,\lq\lq Participatory Sensing Platform Concept for Wildlife Animals in the Himalaya Region, Nepal", Proceedings of 2022 International Conference on Human-Computer Interaction (HCII 2022), 2022.  

\end{査読付}  % end: papers (peer-reviewed)


% \begin{公開}{1}  % open-source urls


% \end{公開}  % end: open-source urls


% \begin{特許}{1}  % patents

% \end{特許}  % end: patents


\begin{発表}{1}  % other talks (Not peer reviewed)

\bibitem{ireneli-7}
Rui Yang, Boming Yang, Aosong Feng, Sixun Ouyang, Moritz Blum, Tianwei She, Yuang Jiang, Freddy Lecue, Jinghui Lu, and Irene Li, \lq\lq Graphusion: A RAG Framework for Knowledge Graph Construction with a Global Perspective,"
International Workshop on Natural Language Processing for Knowledge Graph Construction, WWW, 2025. 

\bibitem{ireneli-8}
Yingjian Chen, Feiyang Li, Xingyu Song, Tianxiao Li, Zixin Xu, Xiujie Chen, Issey Sukeda, and Irene Li,
\lq\lq Exploring the Role of Knowledge Graph-Based RAG in Japanese Medical Question Answering with Small-Scale LLMs," Workshop on Improving Healthcare with Small Language Models, AIME 2025. 


\end{発表}  % end: other talks (Not peer reviewed)


% \begin{報道}{1}  % press (news paper,  televison, etc.)

% \bibitem{ireneli-13}
% (Interview) Gemma Conroy, \textit{How ChatGPT and other AI tools could disrupt scientific publishing} \footnote{https://www.nature.com/articles/d41586-023-03144-w}, \textbf{Nature, Featured News}, 10 October 2023 

% \end{報道}  % end: press (news paper, televison, etc.)



\section{データ科学研究部門 成果要覧}
\begin{招待講演}{1}
%Kuga先生
\bibitem{ykuga458xxxxx}
空閑洋平, mdx: アカデミックHPCクラウドmdxの紹介と今後の技術課題, PCクラスタワークショップ, 2024年6月.

%華井先生
\bibitem{hanai-kyudai}
華井雅俊、 
"ARIM-mdxデータシステム:材料研究向け実験・シミュレーションの統合データプラットフォーム"、
ARIM次世代ナノスケールマテリアル領域 研究会、
2025年3月

\bibitem{hanai-akiba}
華井雅俊、 
"ARIM-mdxデータシステム:材料研究向け実験・シミュレーションの統合データプラットフォーム"、
第19回材料系ワークショップ、
2025年2月

\bibitem{hanai-simpo}
華井雅俊、
"東京大学ARIM データ基盤部門 活動報告",
ARIM「第3回革新的なエネルギー変換を可能とするマテリアル領域」シンポジウム,
2024年12月

\bibitem{hanai-mdx}
華井雅俊、
"ARIM-mdxデータシステム:材料研究向け実験・シミュレーションの統合データプラットフォーム"、
データ活用社会創成シンポジウム2024、
2024年12月

\bibitem{hanai-nbci}
華井雅俊、 
"ARIM-mdx データ収集・保存システムの紹介"、
第2回 NBCI-ARIM 技術交流会、
2024年6月

\end{招待講演}

%\begin{招待論文}{1}

%\end{招待論文}


%\begin{受賞}{1}

%\end{受賞}

% \begin{著書}{1}

% \end{著書}

\begin{雑誌論文}{1}
%小林先生
\bibitem{kobayashi1-1}
Zekun Cai, Renhe Jiang, Xinlei Lian, Chuang Yang, Zhaonan Wang, Zipei Fan, Kota Tsubouchi, Hill Hiroki Kobayashi, Xuan Song, Ryosuke Shibasaki,  "Forecasting Citywide Crowd Transition Process via Convolutional Recurrent Neural Networks", IEEE Transactions on Mobile Computing 23(5) 5433 - 5445.

%河村先生
\bibitem{kawamura_hphi}
K. Ido, M. Kawamura, Y. Motoyama, K. Yoshimi, Y. Yamaji, S. Todo, N. Kawashima, T. Misawa, 
``Update of $\mathcal{H}\Phi$: Newly added functions and methods in versions 2 and 3'',
Comp. Phys. Commun. \textbf{298}, 109093 (2024).

\bibitem{kawamura_pdcu111}
W. Osada, M. Hasegawa, Y. Shiozawa, K. Mukai, S. Yoshimoto, S. Tanaka, M. Kawamura, T. Ozaki and J. Yoshinobu, 
``Chemical process of hydrogen and formic acid on a Pd-deposited Cu(111) surface studied by high-resolution X-ray photoelectron spectroscopy and density functional theory calculations'',
Phys. Chem. Chem. Phys. \textbf{27}, 1978 (2025).

\bibitem{kawamura_wo2}
Y. Muramatsu, D. Hirai, M. Kawamura, S. Minami, Y. Ikeda, T. Shimada, K. Kojima, N. Katayama, K. Takenaka, 
``Topological electronic structure and transport properties of the distorted rutile-type WO$_2$''
APL Mater. \textbf{13}, 011119 (2025).

\bibitem{kawamura_omax}
M. Keivanloo, M. Sandoghchi, M. Reza Mohammadizadeh, M. Kawamura, H. Raebiger, K. Hongo, R. Maezono and M. Khazaei, 
``Superconductivity in o-MAX phases''
Nanoscale, \textbf{17}, 5341 (2025).

\bibitem{kawamura_kmgh3}
S. Lu, R. Akashi, M. Kawamura, S. Tsuneyuki, 
``Assessing the possible superconductivity in doped perovskite hydride KMgH$_3$: Effects of lattice anharmonicity and spin fluctuations''
Phys. Rev. B \textbf{111}, 134516 (2025).

%Li先生
\bibitem{ireneli-1}
Rui Yang, Qingcheng Zeng, Keen You, Yujie Qiao, Lucas Huang, Chia-Chun Hsieh, Benjamin Rosand, Jeremy Goldwasser, Amisha D. Dave, Tiarnan D. L. Keenan, Yuhe Ke, Cheng Hong, Nan Liu, Emily Y. Chew, Dragomir Radev, Zhiyong Lu, Hua Xu, Qingyu Chen, and Irene Li,\lq\lq Ascle—A Python Natural Language Processing Toolkit for Medical Text Generation: Development and Evaluation Study," Journal of Medical Internet Research, vol. 26, 2024.


\bibitem{ireneli-2}
Yuhe Ke, Rui Yang, Sui An Lie, Taylor Xin Yi Lim, Yilin Ning, Irene Li, Hairil Rizal Abdullah, Daniel Shu Wei Ting, and Nan Liu, \lq\lq Mitigating Cognitive Biases in Clinical Decision-Making Through Multi-Agent Conversations Using Large Language Models: Simulation Study,” Journal of Medical Internet Research, vol. 26, 2024.

\bibitem{ireneli-3}
Han Yuan, Mingcheng Zhu, Rui Yang, Han Liu, Irene Li, Chuan Hong, \lq\lq Rethinking Domain-Specific Pretraining by Supervised or Self-Supervised Learning for Chest Radiograph Classification: A Comparative Study Against ImageNet Counterparts in Cold-Start Active Learning," Health Care Science, vol. 3, no. 2, 2025.

\bibitem{ireneli-4}
Dairui Liu, Derek Greene, Irene Li, Xuefei Jiang, and Ruihai Dong, \lq\lq Topic-Centric Explanations for News Recommendation," ACM Transactions on Recommender Systems, vol. 3, no. 2, 2024, Article 9.

\end{雑誌論文}

\begin{査読付}{1}
%小林先生
\bibitem{kobayashi2-1}
Zhuoneng Sui, Haoran Hong, Daisuké Shimotoku, Hill Hiroki Kobayashi, "catAction: Deep learning for enhancing emotional cat-human interactions through the posture-based determination of the degrees of kittens' defensive and offensive aggressions.", Proceedings of the International Conference on Animal-Computer Interaction(ACI) 5-9, 202.  

%鈴村先生
\bibitem{ega}
Toyotaro Suzumura, Hiroki Kanezashi, Shotaro Akahori, "Graph Adapter for Parameter-Efficient Fine-Tuning of EEG Foundation Models", The 39th Annual AAAI Conference on Artificial Intelligence (AAAI-25), The 9th International Workshop on Health Intelligence (W3PHIAI-25), 2025.

\bibitem{gefm}
Limin Wang, Toyotaro Suzumura, Hiroki Kanezashi, "GEFM: Graph-Enhanced EEG Foundation Model", The 39th Annual AAAI Conference on Artificial Intelligence (AAAI-25), Workshop on Large Language Models and Generative AI for Health (GenAI4Health), 2025.

\bibitem{awrs}
 Igor L.R. Azevedo, Toyotaro Suzumura, Yuichiro Yasui, "A Look Into News Avoidance Through AWRS : An Avoidance-Aware Recommender System", Proceedings of the 2025 SIAM International Conference on Data Mining (SDM). Society for Industrial and Applied Mathematics, 2025.

\bibitem{annealing-gnn}
Pablo Loyola, Kento Hasegawa, Andrés Hoyos-Idrobo, Kazuo Ono, Toyotaro Suzumura, Yu Hirate, Masanao Yamaoka, "Annealing Machine-assisted Learning of Graph Neural Network for Combinatorial Optimization", In NeurIPS 2024 Workshop Machine Learning with new Compute Paradigms, 2024.

\bibitem{aln}
Md Mostafizur Rahman, Daisuke Kikuta, Yu Hirate, and Toyotaro Suzumura, "Graph-Based Audience Expansion Model for Marketing Campaigns" In Proceedings of the 47th International ACM SIGIR Conference on Research and Development in Information Retrieval (SIGIR '24). Association for Computing Machinery, New York, NY, USA, 2970–2975, 2024.

\bibitem{p4r}
Chen, Junyi, Toyotaro Suzumura. "A Prompting-Based Representation Learning Method for Recommendation with Large Language Models." The 1st Workshop on Risks, Opportunities, and Evaluation of Generative Models in Recommender Systems (ROEGEN@RECSYS'24), 2024.

%華井先生
\bibitem{hanai-BigData}
Masatoshi Hanai, Mitsuaki Kawamura, Ryo Ishikawa, Toyotaro Suzumura, Kenjiro Taura
"ARIM-mdx Data System: Towards a Nationwide Data Platform for Materials Science"
In Proceedings of the 2024 IEEE International Conference on Big Data (BigData), December, 2024, US.

%河村先生
\bibitem{kawamura_hanai_tone_continuousp}
Y. Tone, M. Hanai, M. Kawamura, K. Taura, T. Suzumura,
``ContinuouSP: Generative Model for Crystal Structure Prediction with Invariance and Continuity'',
 4th Annual AAAI Workshop on AI to Accelerate Science and Engineering (AI2ASE), 2024.
 
%LI先生
\bibitem{ireneli-5}
Dairui Liu, Boming Yang, Honghui Du, Derek Greene, Neil Hurley, Aonghus Lawlor, Ruihai Dong, and Irene Li, \lq \lq RecPrompt: A Self-Tuning Prompting Framework for News Recommendation Using Large Language Models," Proceedings of the 33rd ACM International Conference on Information and Knowledge Management (CIKM), 2024.

\bibitem{ireneli-6}
Yingjian Chen, Haoran Liu, Yinhong Liu, Jinxiang Xie, Rui Yang, Han Yuan, Yanran Fu, Pengyuan Zhou, Qingyu Chen, James Caverlee, and Irene Li, \lq\lq GraphCheck: Breaking Long-Term Text Barriers with Extracted Knowledge Graph-Powered Fact-Checking," Proceedings of the 63rd Annual Meeting of the Association for Computational Linguistics (ACL), 2025.

\end{査読付}

\begin{公開}{1}
%華井先生
\bibitem{hanai-arim-mdx}
"ARIM-mdxデータシステム",
\url{https://arim.mdx.jp/},


\end{公開}

%\begin{特許}{1}

%\end{特許}

\begin{発表}{1}

%宮本先生
\bibitem{dmiya1}
手塚 尚哉, 宮本 大輔, 明石 邦夫, 落合 秀也, ファイルの侵害をフックすることによる ランサムウェアからのデータ保護システム, コンピュータセキュリティシンポジウム, 2024年10月

%川瀬先生
\bibitem{kawase01}
川瀬純也: 配列アライメントを用いた放牧牛の移動行動クラスタリング手法の検討, 動物の行動と管理学会 2024年度研究発表会, 熊本, 2024.9

\bibitem{kawase02}
川瀬純也: 配列アライメントを用いた放牧牛の移動行動クラスタリングの試み, 第33回地理情報システム学会学術研究発表大会, P2-19, 京都, 2024.10

%華井先生
\bibitem{hanai-press}
「材料研究DXを加速するARIM-mdxデータシステムを開発、全国の900名以上が利用開始」,
\url{https://www.u-tokyo.ac.jp/focus/ja/press/z0310_00004.html},

\bibitem{hanai-rxt}
「ハウディ、東京大学との共同研究から生まれたデータ転送IoTデバイス「RxT-01」を販売開始」,
\url{https://prtimes.jp/main/html/rd/p/000000012.000083225.html},

%Kuga先生
\bibitem{ykuga4301yyyyy}
LLM-jp, LLM-jp: A Cross-organizational Project for the Research and Development of Fully Open Japanese LLMs, arXiv preprint, 2024.

\bibitem{ykuga4301xyyyy}
空閑洋平, 中村遼, ソフトウェアメモリを用いたデバイス間データ通信の機能拡張手法の検討, コンピュータシステム・シンポジウム (Comsys), 2024年12月.

%Li先生
\bibitem{ireneli-7}
Rui Yang, Boming Yang, Aosong Feng, Sixun Ouyang, Moritz Blum, Tianwei She, Yuang Jiang, Freddy Lecue, Jinghui Lu, and Irene Li, \lq\lq Graphusion: A RAG Framework for Knowledge Graph Construction with a Global Perspective,"
International Workshop on Natural Language Processing for Knowledge Graph Construction, WWW, 2025. 

\bibitem{ireneli-8}
Yingjian Chen, Feiyang Li, Xingyu Song, Tianxiao Li, Zixin Xu, Xiujie Chen, Issey Sukeda, and Irene Li,
\lq\lq Exploring the Role of Knowledge Graph-Based RAG in Japanese Medical Question Answering with Small-Scale LLMs," Workshop on Improving Healthcare with Small Language Models, AIME 2025. 

\end{発表}

% \begin{特記}{1}

% \end{特記}

%\begin{報道}{1}

%end{報道}


%最後に全員分の成果をマージする。

\end{document}

