\documentclass[11pt]{jarticle}
\usepackage{ITCannual}
\usepackage{amsmath}
\usepackage{amssymb}
\usepackage{times}
\usepackage[dvipdfmx]{graphicx}

\usepackage{url}
%\usepackage[style=numeric]{biblatex}

\title{データ科学研究部門 研究報告}
\author{小林博樹, 鈴村豊太郎, 宮本大輔, 空閑洋平,\\
\textbf{河村光晶, 川瀬純也, 華井雅俊, Li Zihui}}

\begin{document}
\maketitle

\section{データ科学研究部門 概要}
データ科学研究部門では、2023年度、教授3名、准教授3名(特任准教授1名)、講師2名(特任講師2名)、助教3名(特任助教2名)が在籍した。同部門のメンバーは専任教員と特任教員の2つのグループから成る。専任教員はそれぞれが独立して研究活動を行うグループで、特任教員は石川特任教授を中心とする石川グループ研究室である。

% \subsection{専任教員グループの研究テーマ}
%
% \begin{quote}
% \begin{itemize}
% \item 計算機を介した人と生態系のインタラクションの研究(小林)
% \item 大規模グラフニューラルネットワークと様々な実社会問題への応用(鈴村)
% \item データセンタハードウェアへのソフトウェア脆弱試験の適応(空閑)
% \item Title (Kumar)
% \item Title (河村)
% \item データ駆動型知能に基づくアーバンコンピューティング(姜)
% \item 野生動物ワイヤレスセンサネットワーク実証実験基盤構築に向けた研究(川瀬)
% \item グラフニューラルネットワークとその物性予測問題への応用に関する研究(華井)
% \item Title (Zihui)
% \end{itemize}
% \end{quote}

\section{データ科学研究部門 教員研究活動}

\input{Kobayashi/ITCannual-child-Kobayashi}
 \begin{雑誌論文}{1}
 \bibitem{kobayashi1-1}
Zekun Cai, Renhe Jiang, Xinlei Lian, Chuang Yang, Zhaonan Wang, Zipei Fan, Kota Tsubouchi, Hill Hiroki Kobayashi, Xuan Song, Ryosuke Shibasaki,  "Forecasting Citywide Crowd Transition Process via Convolutional Recurrent Neural Networks", IEEE Transactions on Mobile Computing 23(5) 5433 - 5445.


 \end{雑誌論文}

 \begin{査読付}{1}
 \bibitem{kobayashi2-1}
Zhuoneng Sui, Haoran Hong, Daisuké Shimotoku, Hill Hiroki Kobayashi, "catAction: Deep learning for enhancing emotional cat-human interactions through the posture-based determination of the degrees of kittens' defensive and offensive aggressions.", Proceedings of the International Conference on Animal-Computer Interaction(ACI) 5-9, 202.  


 \end{査読付}



\input{Suzumura/ITCannual-child-Suzumura}
\input{Suzumura/ITCannual-list-Suzumura}

\input{Miyamoto/ITCannual-child-Miyamoto}
%\begin{発表}{1}
%
%\bibitem{dmiya1}
%手塚 尚哉, 宮本 大輔, 明石 邦夫, 落合 秀也, ファイルの侵害をフックすることによる ランサムウェアからのデータ保護システム, コンピュータセキュリティシンポジウム, 2024年10月
%
%\end{発表}


\subsection{データセンタハードウェアへのソフトウェア脆弱試験の適応(空閑 洋平)}

現在のデータセンタ環境では、AIモデル作成等を高速化するためにGPU等専用アクセラレータが広く使用されている。
専用アクセラレータを用いた計算環境は、既存のCPUを中心に構成されていたクラウド型の仮想マシンクラスタから、CPUをバイパスしてデバイス間で直接データ通信するヘテロジニアス構成に移行したことで、システム全体のブラックボックス化が進んでいる。
今後、専用アクセラレータを中心としたデータセンタ環境では、CPUをバイパスするデバイス間通信が増加することで、セキュリティ監視や脆弱性試験、管理手法、データ通信内容の可視化手法といった、クラウド型クラスタ環境で実施している運用課題が顕在化すると考えられる。
本年度は、昨年度に引き続き、PIM (Processing in Memory)型のデバイスメモリの設計開発を実施し、PoCを用いたLinux NVMeドライバの既存の脆弱性の再現手法を開発し、情報処理学会コンピュータシステム・シンポジウム2024で報告した\cite{ykuga4301xyyyy}。

その他の成果としては、NIIが主催するLLM勉強会の活動に参加し、172Bモデル作成に関する活動内容を報告した\cite{ykuga4301yyyyy}。
また、クラスタワークショップ in すずかけ台2024に参加し、mdxのRDMAネットワークに関する招待講演を実施した\cite{ykuga458xxxxx}。

\begin{招待講演}{1}
\bibitem{ykuga458xxxxx}
空閑洋平, mdx: アカデミックHPCクラウドmdxの紹介と今後の技術課題, PCクラスタワークショップ, 2024年6月.

\end{招待講演}

\begin{発表}{1}
\bibitem{ykuga4301yyyyy}
LLM-jp, LLM-jp: A Cross-organizational Project for the Research and Development of Fully Open Japanese LLMs, arXiv preprint, 2024.

\bibitem{ykuga4301xyyyy}
空閑洋平, 中村遼, ソフトウェアメモリを用いたデバイス間データ通信の機能拡張手法の検討, コンピュータシステム・シンポジウム (Comsys), 2024年12月.

\end{発表}


\subsection{第一原理計算とデータ科学・機械学習による物質科学研究(河村 光晶)}

物質を構成する電子や原子に対して、基本法則となる(相対論的)量子力学や統計物理学に基づく理論的研究と、様々な環境(温度、圧力等)やプローブ(電子線、可視光、X線、中性子線、物理量測定等)における実験的研究の両者を協調的に行う事で、既存の現象の理解や新たなる物質探索をより効果的に進めることができる。
実験との比較を行うにあたり、物性物理学の理論を多様な組成$\cdot$構造を持つ現実の物質に適用するためには計算機によるシミュレーションが不可欠となり、スパコンやmdxのような高速$\cdot$大規模な並列計算資源およびデータ格納環境が利用される。
我々はそのような大規模計算機における物質科学シミュレーションの手法やプログラムの開発、およびそれを実際の物質$\cdot$現象に適用する研究を行っている。
%利根さん
~\cite{kawamura_hanai_tone_continuousp}
%HPHi
~\cite{kawamura_hphi}
%WO$_2$
~\cite{kawamura_wo2}
%KMgH$_3$
~\cite{kawamura_kmgh3}
%o-MAX相
~\cite{kawamura_omax}
%Pd-deposited Cu(111) surface
~\cite{kawamura_pdcu111}
\begin{招待講演}{1}
\end{招待講演}

\begin{招待論文}{1}
\end{招待論文}

\begin{受賞}{1}
\end{受賞}


\begin{雑誌論文}{4}
\bibitem{kawamura_la2ios2}
H. Ishikawa, T. Yajima, D. Nishio-Hamane, S. Imajo, K. Kindo, and M. Kawamura,
``Superconductivity at 12 K in ${\mathrm{La}}_{2}{\mathrm{IOs}}_{2}$: A $5d$ metal with osmium honeycomb layer'',
Physical Review Materials \textbf{7}, 054804 (2023).

\end{雑誌論文}



\subsection{野生動物ワイヤレスセンサネットワークと時空間行動分析に関する研究(川瀬 純也)}
本研究室では、野生動物装着型ワイヤレスセンサーネットワーク機構による自然環境下でのデータ収集手法の開発と、それによって得られるデータの解析手法についての研究を行っている。人間が容易には侵入できないエリアでの継続的なデータ収集機構として種々の社会問題の解決に寄与することを目指している。

野生動物を対象に含む時空間行動分析の手法として、配列アライメント手法を用いた類型化手法に着目し、研究を進めている。野生動物の時空間行動は、その意図や目的などを明示的に把握することができず、GPSデータなどの移動データから推測するしかない。そのGPSデータも自然環境下では測位精度が低く、野生動物装着型デバイスのバッテリー持続期間の問題からも、測位の間隔やタイミングが整ったデータを収集することは困難である。多種多様で、大量かつ欠落部分を含む移動データを分析する上で、これらの問題点を考慮した定量的な類型化手法は非常に重要となる。

2024年度は、2023年度から引き続き北海道の広い放牧地で自由に移動し活動する乳牛のGPSデータ等の収集を行い、6か月に渡る継続的な乳牛のGPSデータを収集することができた。また2023年度に収集した同様の乳牛のGPSデータと合わせて、これらを用いて類型化手法の検討を進めてきた。観光地などでの人の移動行動を対象とした類型化の既存手法では、対象エリア内を機能や空間的なつながりにもとづいて分割し、類型化に用いる。しかしこの手法は、隔たりのない自然環境を自由に移動する動物の移動行動には適用できないと考えられた。そこで本研究では実際の乳牛たちの移動行動の特徴にもとづいて、類型化の単位(ここでは1日24時間)ごとに対象エリアの分割を行えるようにした。また、1日毎の類型化結果をもとに、分析期間(約4カ月間)を通した類型化を行った。これにより、乳牛たちの日々の移動行動の類似性と、分析期間を通したその類似性の変化をもとに、移動行動の類型化を行うことが可能であることが示された。以上については、動物行動関係学会と地理情報科学関係学会の2カ所で報告を行った。\cite{kawase01, kawase02}また、2024年度に収集したGPSデータも合わせて対象とし、手法の妥当性の検討を進めている。

さらに、遅延耐性ネットワーク (DTN)技術を用いたワイヤレスセンサーネットワークのシミュレーションを行い、被覆面積の効率的な最大化手法について検討している。DTN技術を用いた野生動物装着型ワイヤレスセンサーネットワークにおいては、異なる群れの間を行き来したり、積極的に他の個体と接触したりする個体の存在が重要となる。そのような個体が、広くデータを伝播させる役割を担うことができると考えられるからである。類型化手法においては、「一緒に行動する群れ」を特定するだけでなく、「群れの間を行き来するような特徴的な少数派」を効率的に見つけ出すことを目的のひとつとしている。これらに着目し、継続的に研究を行っていく。
\begin{発表}{1}

\bibitem{kawase01}
川瀬純也: 配列アライメントを用いた放牧牛の移動行動クラスタリング手法の検討, 動物の行動と管理学会 2024年度研究発表会, 熊本, 2024.9

\bibitem{kawase02}
川瀬純也: 配列アライメントを用いた放牧牛の移動行動クラスタリングの試み, 第33回地理情報システム学会学術研究発表大会, P2-19, 京都, 2024.10

\end{発表}

\input{Hanai/ITCannual-child-Hanai}
\begin{招待講演}{5}

\bibitem{hanai-kyudai}
華井雅俊、 
"ARIM-mdxデータシステム:材料研究向け実験・シミュレーションの統合データプラットフォーム"、
ARIM次世代ナノスケールマテリアル領域 研究会、
2025年3月

\bibitem{hanai-akiba}
華井雅俊、 
"ARIM-mdxデータシステム:材料研究向け実験・シミュレーションの統合データプラットフォーム"、
第19回材料系ワークショップ、
2025年2月

\bibitem{hanai-simpo}
華井雅俊、
"東京大学ARIM データ基盤部門 活動報告",
ARIM「第3回革新的なエネルギー変換を可能とするマテリアル領域」シンポジウム,
2024年12月


\bibitem{hanai-mdx}
華井雅俊、
"ARIM-mdxデータシステム:材料研究向け実験・シミュレーションの統合データプラットフォーム"、
データ活用社会創成シンポジウム2024、
2024年12月

\bibitem{hanai-nbci}
華井雅俊、 
"ARIM-mdx データ収集・保存システムの紹介"、
第2回 NBCI-ARIM 技術交流会、
2024年6月

\end{招待講演}

\begin{査読付}{3}

\bibitem{hanai-BigData}
Masatoshi Hanai, Mitsuaki Kawamura, Ryo Ishikawa, Toyotaro Suzumura, Kenjiro Taura
"ARIM-mdx Data System: Towards a Nationwide Data Platform for Materials Science"
In Proceedings of the 2024 IEEE International Conference on Big Data (BigData), December, 2024, US.


% \bibitem{hanai-UCC}
% Masatoshi Hanai, Mitsuaki Kawamura, Ryo Ishikawa, Toyotaro Suzumura, Kenjiro Taura
% "Cloud Data Acquisition from Shared-Use Facilities in A University-Scale Laboratory Information Management System."
% In Proceedings of the 16th IEEE/ACM International Conference on Utility and Cloud Computing (UCC 2023), December 5, 2023, Taormina.

% \bibitem{hanai-xiaohang}
% Xiaohang Xu, Toyotaro Suzumura, Jiawei Yong, Masatoshi Hanai, Chuang Yang, Hiroki Kanezashi, Renhe Jiang, Shintaro Fukushima, 
% "Revisiting Mobility Modeling with Graph: A Graph Transformer Model for Next Point-of-Interest Recommendation."
% In Proceedings of the 31st ACM International Conference on Advances in Geographic Information Systems (SIGSPATIAL ’23). Association for Computing Machinery, New York, NY, USA, Article
% 94, 1–10.
% \bibitem{job-jsai}
% 脇聡志, 鈴村豊太郎, 金刺宏樹, 華井雅俊, 小林秀. "強化学習によるマッチング数を最大化するジョブ推薦システム." 人工知能学会全国大会論文集 第 37 回 (2023), 一般社団法人 人工知能学会, 2023.

\end{査読付}

\begin{発表}{2}
\bibitem{hanai-press}
「材料研究DXを加速するARIM-mdxデータシステムを開発、全国の900名以上が利用開始」,
\url{https://www.u-tokyo.ac.jp/focus/ja/press/z0310_00004.html},

\bibitem{hanai-rxt}
「ハウディ、東京大学との共同研究から生まれたデータ転送IoTデバイス「RxT-01」を販売開始」,
\url{https://prtimes.jp/main/html/rd/p/000000012.000083225.html},

\end{発表}

\begin{公開}{1}
\bibitem{hanai-arim-mdx}
"ARIM-mdxデータシステム",
\url{https://arim.mdx.jp/},
\end{公開}


\subsection{大規模言語モデルの研究とその応用(Li Zihui)}

% Educational Large Language Models (LLMs): This research area focuses on enhancing the discovery of educational resources and the interpretability of transformer models in NLP education. We explored whether traditional methods could effectively identify high-quality study materials and proposed a transfer learning-based pipeline to improve resource discovery, especially for generating introductory paragraphs in educational content \cite{ireneli-3,ireneli-9}. Additionally, we assessed the potential of LLMs as tools for learning support, presenting a benchmark to evaluate their performance in various NLP tasks \cite{ireneli-5}. Our investigation also included the generation of concise survey articles in the NLP domain using LLMs. While GPT-created surveys were found to be more up-to-date and accessible than human-written ones, some limitations, such as occasional factual inaccuracies, were noted \cite{ireneli-11}. Furthermore, we explored the use of LLMs in teaching complex legal concepts through narrative methods \cite{ireneli-8}. Our LLM research works \cite{ireneli-1} have been noticed by \textbf{Nature News}, and I had the opportunity to be interviewed by a reporter, where I shared my insights on the impact of Large Language Models (LLMs) from the academia perspective \cite{ireneli-13}.

% Medical Large Language Models (LLMs): In medical LLM research, we focus on enhancing LLMs' capabilities in medical question answering and teaching complex legal concepts via storytelling. Collaborating with MatsuoLab, we developed a framework combining knowledge graphs and ranking techniques to boost LLMs' effectiveness in medical queries \cite{ireneli-7}. We also introduced Ascle, a Python NLP toolkit for medical text generation \cite{ireneli-10}.

% Other Benchmarks for NLP Open Questions: Our research extends to improving knowledge graph completion, evaluating LLMs in NLP problem-solving, and developing medical text generation tools. We investigated the enhancement of knowledge graph completion using node neighborhood data \cite{ireneli-4} and explored methods to better interpret transformer models by emphasizing crucial information \cite{ireneli-6}. Besides, we have been investigating other branches including graph methods for news encoding \cite{ireneli-2,ireneli-12}.


This year’s research focused on three main areas: advancing clinical NLP tools and decision support, developing explainable and adaptive recommendation systems, and improving factual reasoning in language models using knowledge graphs. Across journals, conferences, and workshops, these works contributed novel methods and practical frameworks for enhancing AI reliability, personalization, and interpretability in healthcare and information systems.

In the medical AI domain, Ascle was introduced as an open-source NLP toolkit for medical text generation, offering fine-tuned models and user-friendly APIs to support structured content creation in clinical settings \cite{ireneli-1}. A simulation study demonstrated the effectiveness of multi-agent LLM conversations in reducing cognitive biases in clinical decision-making. Additionally, a comparative analysis of pretraining methods for chest radiograph classification showed that task-specific self-supervised models outperform standard baselines in low-label regimes\cite{ireneli-2,ireneli-3}.

For recommendation systems, one study proposed topic-centric explanations to improve the transparency of news recommendations by linking articles to user-relevant topic distributions \cite{ireneli-4}. Another work introduced RecPrompt, a self-tuning prompting framework that leverages LLMs to dynamically optimize news recommendations based on user preferences, achieving strong performance without additional model training \cite{ireneli-5}.

In the area of factual reasoning, GraphCheck presented a fact-checking framework that uses knowledge graphs extracted from text to support long-context verification \cite{ireneli-6}, particularly effective in medical and general domains. Graphusion proposed a retrieval-augmented generation (RAG) framework for zero-shot knowledge graph construction, supporting downstream tasks like subgraph completion. A related study applied knowledge graph-based RAG to Japanese medical QA, highlighting both the promise and current limitations in low-resource, domain-specific applications \cite{ireneli-7,ireneli-8}.

% \begin{雑誌論文}{1}

% \bibitem{sample}
% Irene Li, Jessica Pan, Jeremy Goldwasser, Neha Verma, Wai Pan Wong, Muhammed Yavuz Nuzumlalı, Benjamin Rosand, Yixin Li, Matthew Zhang, David Chang, R. Andrew Taylor, Harlan M. Krumholz, Dragomir Radev,\lq\lq Neural Natural Language Processing for unstructured data in electronic health records: A review", Computer Science Review, volume 46, 2022.

% \end{雑誌論文}

% \begin{査読付}{1}

% \bibitem{feng2022diffuser}
% Aosong Feng and Irene Li and Yuang Jiang andRex  Ying,\lq\lq Diffuser: Efficient Transformers with Multi-hop Attention Diffusion for Long Sequences", Proceedings of Thirty-Seventh AAAI Conference on Artificial Intelligence (AAAI), 2023.

% \end{査読付}

% \begin{発表}{1}

% \bibitem{li2023nnkgc}
% Zihui Li and Boming Yang and Toyotaro Suzumura,\lq\lq NNKGC: Improving Knowledge Graph Completion with Node Neighborhoods", arXiv preprint, 2023.

% \end{発表}

% \begin{招待講演}{1}  % invited talks

% sample
% \bibitem{sample-kobayashi3-1}
% Hill Hiroki Kobayashi, mdx: A Cloud Platform for Supporting Data Science and Cross-Disciplinary Research Collaborations, the Nepal JSPS Alumni Association (NJAA), hosted its 7th Symposium, 29 November, 2022.

% \bibitem{ireneli-1}
% Irene Li, A Journey from Transformers to Large Language Models: an Educational Perspective, 2023 the 1st International Conference on AI-generated Content (AIGC2023), Aug, 2023


% \end{招待講演}  % end: invited talks

% \begin{招待論文}{1}  % invited papers



% \end{招待論文}  % end: invited papers


% \begin{受賞}{1}  % awards

% \bibitem{ireneli-2}
% Boming Yang, Dairui Liu, Toyotaro Suzumura, Ruihai Dong and Irene Li,\lq\lq Going Beyond Local: Global Graph-Enhanced Personalized News Recommendations", Proceedings of the 17th ACM Conference on Recommender Systems  (RecSys 2023), 2023 (Best Student Paper Award)

% \end{受賞}  % end: awards


% \begin{著書}{1}  % books

% \end{著書}  % end: books


\begin{雑誌論文}{1}  % journals

% % sample
% % \bibitem{sample-kobayashi1-3}
% % Wenjing Li, Haoran Zhang, Jinyu Chen, Peiran Li, Yuhao Yao, Xiaodan Shi,  Mariko Shibasaki, Hill Hiroki Kobayashi, Xuan Song and Ryosuke Shibasaki, \lq\lq Metagraph-Based Life Pattern Clustering With Big Human Mobility Data", IEEE Transactions on Big Data, Feb, 2023.

\bibitem{ireneli-1}
Rui Yang, Qingcheng Zeng, Keen You, Yujie Qiao, Lucas Huang, Chia-Chun Hsieh, Benjamin Rosand, Jeremy Goldwasser, Amisha D. Dave, Tiarnan D. L. Keenan, Yuhe Ke, Cheng Hong, Nan Liu, Emily Y. Chew, Dragomir Radev, Zhiyong Lu, Hua Xu, Qingyu Chen, and Irene Li,\lq\lq Ascle—A Python Natural Language Processing Toolkit for Medical Text Generation: Development and Evaluation Study," Journal of Medical Internet Research, vol. 26, 2024.


\bibitem{ireneli-2}
Yuhe Ke, Rui Yang, Sui An Lie, Taylor Xin Yi Lim, Yilin Ning, Irene Li, Hairil Rizal Abdullah, Daniel Shu Wei Ting, and Nan Liu, \lq\lq Mitigating Cognitive Biases in Clinical Decision-Making Through Multi-Agent Conversations Using Large Language Models: Simulation Study,” Journal of Medical Internet Research, vol. 26, 2024.

\bibitem{ireneli-3}
Han Yuan, Mingcheng Zhu, Rui Yang, Han Liu, Irene Li, Chuan Hong, \lq\lq Rethinking Domain-Specific Pretraining by Supervised or Self-Supervised Learning for Chest Radiograph Classification: A Comparative Study Against ImageNet Counterparts in Cold-Start Active Learning," Health Care Science, vol. 3, no. 2, 2025.

\bibitem{ireneli-4}
Dairui Liu, Derek Greene, Irene Li, Xuefei Jiang, and Ruihai Dong, \lq\lq Topic-Centric Explanations for News Recommendation," ACM Transactions on Recommender Systems, vol. 3, no. 2, 2024, Article 9.

\end{雑誌論文}  % end: journals


\begin{査読付}{1}  % papers (peer-reviewed)

\bibitem{ireneli-5}
Dairui Liu, Boming Yang, Honghui Du, Derek Greene, Neil Hurley, Aonghus Lawlor, Ruihai Dong, and Irene Li, \lq \lq RecPrompt: A Self-Tuning Prompting Framework for News Recommendation Using Large Language Models," Proceedings of the 33rd ACM International Conference on Information and Knowledge Management (CIKM), 2024.

\bibitem{ireneli-6}
Yingjian Chen, Haoran Liu, Yinhong Liu, Jinxiang Xie, Rui Yang, Han Yuan, Yanran Fu, Pengyuan Zhou, Qingyu Chen, James Caverlee, and Irene Li, \lq\lq GraphCheck: Breaking Long-Term Text Barriers with Extracted Knowledge Graph-Powered Fact-Checking," Proceedings of the 63rd Annual Meeting of the Association for Computational Linguistics (ACL), 2025.

% sample
% \bibitem{sample-kobayashi2-1}
% Daisuk\'e Shimotoku, Tian Yuan, Laxmi Kumar Parajuli and Hill Hiroki Kobayashi,\lq\lq Participatory Sensing Platform Concept for Wildlife Animals in the Himalaya Region, Nepal", Proceedings of 2022 International Conference on Human-Computer Interaction (HCII 2022), 2022.  

\end{査読付}  % end: papers (peer-reviewed)


% \begin{公開}{1}  % open-source urls


% \end{公開}  % end: open-source urls


% \begin{特許}{1}  % patents

% \end{特許}  % end: patents


\begin{発表}{1}  % other talks (Not peer reviewed)

\bibitem{ireneli-7}
Rui Yang, Boming Yang, Aosong Feng, Sixun Ouyang, Moritz Blum, Tianwei She, Yuang Jiang, Freddy Lecue, Jinghui Lu, and Irene Li, \lq\lq Graphusion: A RAG Framework for Knowledge Graph Construction with a Global Perspective,"
International Workshop on Natural Language Processing for Knowledge Graph Construction, WWW, 2025. 

\bibitem{ireneli-8}
Yingjian Chen, Feiyang Li, Xingyu Song, Tianxiao Li, Zixin Xu, Xiujie Chen, Issey Sukeda, and Irene Li,
\lq\lq Exploring the Role of Knowledge Graph-Based RAG in Japanese Medical Question Answering with Small-Scale LLMs," Workshop on Improving Healthcare with Small Language Models, AIME 2025. 


\end{発表}  % end: other talks (Not peer reviewed)


% \begin{報道}{1}  % press (news paper,  televison, etc.)

% \bibitem{ireneli-13}
% (Interview) Gemma Conroy, \textit{How ChatGPT and other AI tools could disrupt scientific publishing} \footnote{https://www.nature.com/articles/d41586-023-03144-w}, \textbf{Nature, Featured News}, 10 October 2023 

% \end{報道}  % end: press (news paper, televison, etc.)



\section{データ科学研究部門 成果要覧}
%\begin{招待講演}{1}
%Kuga先生
\bibitem{ykuga458xxxxx}
空閑洋平, mdx: アカデミックHPCクラウドmdxの紹介と今後の技術課題, PCクラスタワークショップ, 2024年6月.

%華井先生
\bibitem{hanai-kyudai}
華井雅俊、 
"ARIM-mdxデータシステム:材料研究向け実験・シミュレーションの統合データプラットフォーム"、
ARIM次世代ナノスケールマテリアル領域 研究会、
2025年3月

\bibitem{hanai-akiba}
華井雅俊、 
"ARIM-mdxデータシステム:材料研究向け実験・シミュレーションの統合データプラットフォーム"、
第19回材料系ワークショップ、
2025年2月

\bibitem{hanai-simpo}
華井雅俊、
"東京大学ARIM データ基盤部門 活動報告",
ARIM「第3回革新的なエネルギー変換を可能とするマテリアル領域」シンポジウム,
2024年12月

\bibitem{hanai-mdx}
華井雅俊、
"ARIM-mdxデータシステム:材料研究向け実験・シミュレーションの統合データプラットフォーム"、
データ活用社会創成シンポジウム2024、
2024年12月

\bibitem{hanai-nbci}
華井雅俊、 
"ARIM-mdx データ収集・保存システムの紹介"、
第2回 NBCI-ARIM 技術交流会、
2024年6月

\end{招待講演}

%\begin{招待論文}{1}

%\end{招待論文}


%\begin{受賞}{1}

%\end{受賞}

% \begin{著書}{1}

% \end{著書}

\begin{雑誌論文}{1}
%小林先生
\bibitem{kobayashi1-1}
Zekun Cai, Renhe Jiang, Xinlei Lian, Chuang Yang, Zhaonan Wang, Zipei Fan, Kota Tsubouchi, Hill Hiroki Kobayashi, Xuan Song, Ryosuke Shibasaki,  "Forecasting Citywide Crowd Transition Process via Convolutional Recurrent Neural Networks", IEEE Transactions on Mobile Computing 23(5) 5433 - 5445.

%河村先生
\bibitem{kawamura_hphi}
K. Ido, M. Kawamura, Y. Motoyama, K. Yoshimi, Y. Yamaji, S. Todo, N. Kawashima, T. Misawa, 
``Update of $\mathcal{H}\Phi$: Newly added functions and methods in versions 2 and 3'',
Comp. Phys. Commun. \textbf{298}, 109093 (2024).

\bibitem{kawamura_pdcu111}
W. Osada, M. Hasegawa, Y. Shiozawa, K. Mukai, S. Yoshimoto, S. Tanaka, M. Kawamura, T. Ozaki and J. Yoshinobu, 
``Chemical process of hydrogen and formic acid on a Pd-deposited Cu(111) surface studied by high-resolution X-ray photoelectron spectroscopy and density functional theory calculations'',
Phys. Chem. Chem. Phys. \textbf{27}, 1978 (2025).

\bibitem{kawamura_wo2}
Y. Muramatsu, D. Hirai, M. Kawamura, S. Minami, Y. Ikeda, T. Shimada, K. Kojima, N. Katayama, K. Takenaka, 
``Topological electronic structure and transport properties of the distorted rutile-type WO$_2$''
APL Mater. \textbf{13}, 011119 (2025).

\bibitem{kawamura_omax}
M. Keivanloo, M. Sandoghchi, M. Reza Mohammadizadeh, M. Kawamura, H. Raebiger, K. Hongo, R. Maezono and M. Khazaei, 
``Superconductivity in o-MAX phases''
Nanoscale, \textbf{17}, 5341 (2025).

\bibitem{kawamura_kmgh3}
S. Lu, R. Akashi, M. Kawamura, S. Tsuneyuki, 
``Assessing the possible superconductivity in doped perovskite hydride KMgH$_3$: Effects of lattice anharmonicity and spin fluctuations''
Phys. Rev. B \textbf{111}, 134516 (2025).

%Li先生
\bibitem{ireneli-1}
Rui Yang, Qingcheng Zeng, Keen You, Yujie Qiao, Lucas Huang, Chia-Chun Hsieh, Benjamin Rosand, Jeremy Goldwasser, Amisha D. Dave, Tiarnan D. L. Keenan, Yuhe Ke, Cheng Hong, Nan Liu, Emily Y. Chew, Dragomir Radev, Zhiyong Lu, Hua Xu, Qingyu Chen, and Irene Li,\lq\lq Ascle—A Python Natural Language Processing Toolkit for Medical Text Generation: Development and Evaluation Study," Journal of Medical Internet Research, vol. 26, 2024.


\bibitem{ireneli-2}
Yuhe Ke, Rui Yang, Sui An Lie, Taylor Xin Yi Lim, Yilin Ning, Irene Li, Hairil Rizal Abdullah, Daniel Shu Wei Ting, and Nan Liu, \lq\lq Mitigating Cognitive Biases in Clinical Decision-Making Through Multi-Agent Conversations Using Large Language Models: Simulation Study,” Journal of Medical Internet Research, vol. 26, 2024.

\bibitem{ireneli-3}
Han Yuan, Mingcheng Zhu, Rui Yang, Han Liu, Irene Li, Chuan Hong, \lq\lq Rethinking Domain-Specific Pretraining by Supervised or Self-Supervised Learning for Chest Radiograph Classification: A Comparative Study Against ImageNet Counterparts in Cold-Start Active Learning," Health Care Science, vol. 3, no. 2, 2025.

\bibitem{ireneli-4}
Dairui Liu, Derek Greene, Irene Li, Xuefei Jiang, and Ruihai Dong, \lq\lq Topic-Centric Explanations for News Recommendation," ACM Transactions on Recommender Systems, vol. 3, no. 2, 2024, Article 9.

\end{雑誌論文}

\begin{査読付}{1}
%小林先生
\bibitem{kobayashi2-1}
Zhuoneng Sui, Haoran Hong, Daisuké Shimotoku, Hill Hiroki Kobayashi, "catAction: Deep learning for enhancing emotional cat-human interactions through the posture-based determination of the degrees of kittens' defensive and offensive aggressions.", Proceedings of the International Conference on Animal-Computer Interaction(ACI) 5-9, 202.  

%鈴村先生
\bibitem{ega}
Toyotaro Suzumura, Hiroki Kanezashi, Shotaro Akahori, "Graph Adapter for Parameter-Efficient Fine-Tuning of EEG Foundation Models", The 39th Annual AAAI Conference on Artificial Intelligence (AAAI-25), The 9th International Workshop on Health Intelligence (W3PHIAI-25), 2025.

\bibitem{gefm}
Limin Wang, Toyotaro Suzumura, Hiroki Kanezashi, "GEFM: Graph-Enhanced EEG Foundation Model", The 39th Annual AAAI Conference on Artificial Intelligence (AAAI-25), Workshop on Large Language Models and Generative AI for Health (GenAI4Health), 2025.

\bibitem{awrs}
 Igor L.R. Azevedo, Toyotaro Suzumura, Yuichiro Yasui, "A Look Into News Avoidance Through AWRS : An Avoidance-Aware Recommender System", Proceedings of the 2025 SIAM International Conference on Data Mining (SDM). Society for Industrial and Applied Mathematics, 2025.

\bibitem{annealing-gnn}
Pablo Loyola, Kento Hasegawa, Andrés Hoyos-Idrobo, Kazuo Ono, Toyotaro Suzumura, Yu Hirate, Masanao Yamaoka, "Annealing Machine-assisted Learning of Graph Neural Network for Combinatorial Optimization", In NeurIPS 2024 Workshop Machine Learning with new Compute Paradigms, 2024.

\bibitem{aln}
Md Mostafizur Rahman, Daisuke Kikuta, Yu Hirate, and Toyotaro Suzumura, "Graph-Based Audience Expansion Model for Marketing Campaigns" In Proceedings of the 47th International ACM SIGIR Conference on Research and Development in Information Retrieval (SIGIR '24). Association for Computing Machinery, New York, NY, USA, 2970–2975, 2024.

\bibitem{p4r}
Chen, Junyi, Toyotaro Suzumura. "A Prompting-Based Representation Learning Method for Recommendation with Large Language Models." The 1st Workshop on Risks, Opportunities, and Evaluation of Generative Models in Recommender Systems (ROEGEN@RECSYS'24), 2024.

%華井先生
\bibitem{hanai-BigData}
Masatoshi Hanai, Mitsuaki Kawamura, Ryo Ishikawa, Toyotaro Suzumura, Kenjiro Taura
"ARIM-mdx Data System: Towards a Nationwide Data Platform for Materials Science"
In Proceedings of the 2024 IEEE International Conference on Big Data (BigData), December, 2024, US.

%河村先生
\bibitem{kawamura_hanai_tone_continuousp}
Y. Tone, M. Hanai, M. Kawamura, K. Taura, T. Suzumura,
``ContinuouSP: Generative Model for Crystal Structure Prediction with Invariance and Continuity'',
 4th Annual AAAI Workshop on AI to Accelerate Science and Engineering (AI2ASE), 2024.
 
%LI先生
\bibitem{ireneli-5}
Dairui Liu, Boming Yang, Honghui Du, Derek Greene, Neil Hurley, Aonghus Lawlor, Ruihai Dong, and Irene Li, \lq \lq RecPrompt: A Self-Tuning Prompting Framework for News Recommendation Using Large Language Models," Proceedings of the 33rd ACM International Conference on Information and Knowledge Management (CIKM), 2024.

\bibitem{ireneli-6}
Yingjian Chen, Haoran Liu, Yinhong Liu, Jinxiang Xie, Rui Yang, Han Yuan, Yanran Fu, Pengyuan Zhou, Qingyu Chen, James Caverlee, and Irene Li, \lq\lq GraphCheck: Breaking Long-Term Text Barriers with Extracted Knowledge Graph-Powered Fact-Checking," Proceedings of the 63rd Annual Meeting of the Association for Computational Linguistics (ACL), 2025.

\end{査読付}

\begin{公開}{1}
%華井先生
\bibitem{hanai-arim-mdx}
"ARIM-mdxデータシステム",
\url{https://arim.mdx.jp/},


\end{公開}

%\begin{特許}{1}

%\end{特許}

\begin{発表}{1}

%宮本先生
\bibitem{dmiya1}
手塚 尚哉, 宮本 大輔, 明石 邦夫, 落合 秀也, ファイルの侵害をフックすることによる ランサムウェアからのデータ保護システム, コンピュータセキュリティシンポジウム, 2024年10月

%川瀬先生
\bibitem{kawase01}
川瀬純也: 配列アライメントを用いた放牧牛の移動行動クラスタリング手法の検討, 動物の行動と管理学会 2024年度研究発表会, 熊本, 2024.9

\bibitem{kawase02}
川瀬純也: 配列アライメントを用いた放牧牛の移動行動クラスタリングの試み, 第33回地理情報システム学会学術研究発表大会, P2-19, 京都, 2024.10

%華井先生
\bibitem{hanai-press}
「材料研究DXを加速するARIM-mdxデータシステムを開発、全国の900名以上が利用開始」,
\url{https://www.u-tokyo.ac.jp/focus/ja/press/z0310_00004.html},

\bibitem{hanai-rxt}
「ハウディ、東京大学との共同研究から生まれたデータ転送IoTデバイス「RxT-01」を販売開始」,
\url{https://prtimes.jp/main/html/rd/p/000000012.000083225.html},

%Kuga先生
\bibitem{ykuga4301yyyyy}
LLM-jp, LLM-jp: A Cross-organizational Project for the Research and Development of Fully Open Japanese LLMs, arXiv preprint, 2024.

\bibitem{ykuga4301xyyyy}
空閑洋平, 中村遼, ソフトウェアメモリを用いたデバイス間データ通信の機能拡張手法の検討, コンピュータシステム・シンポジウム (Comsys), 2024年12月.

%Li先生
\bibitem{ireneli-7}
Rui Yang, Boming Yang, Aosong Feng, Sixun Ouyang, Moritz Blum, Tianwei She, Yuang Jiang, Freddy Lecue, Jinghui Lu, and Irene Li, \lq\lq Graphusion: A RAG Framework for Knowledge Graph Construction with a Global Perspective,"
International Workshop on Natural Language Processing for Knowledge Graph Construction, WWW, 2025. 

\bibitem{ireneli-8}
Yingjian Chen, Feiyang Li, Xingyu Song, Tianxiao Li, Zixin Xu, Xiujie Chen, Issey Sukeda, and Irene Li,
\lq\lq Exploring the Role of Knowledge Graph-Based RAG in Japanese Medical Question Answering with Small-Scale LLMs," Workshop on Improving Healthcare with Small Language Models, AIME 2025. 

\end{発表}

% \begin{特記}{1}

% \end{特記}

%\begin{報道}{1}

%end{報道}


%最後に全員分の成果をマージする。

\end{document}

