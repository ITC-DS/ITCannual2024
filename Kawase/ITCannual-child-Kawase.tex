\subsection{野生動物ワイヤレスセンサネットワークと時空間行動分析に関する研究(川瀬 純也)}
本研究室では、野生動物装着型ワイヤレスセンサーネットワーク機構による自然環境下でのデータ収集手法の開発と、それによって得られるデータの解析手法についての研究を行っている。人間が容易には侵入できないエリアでの継続的なデータ収集機構として種々の社会問題の解決に寄与することを目指している。

野生動物を対象に含む時空間行動分析の手法として、配列アライメント手法を用いた類型化手法に着目し、研究を進めている。野生動物の時空間行動は、その意図や目的などを明示的に把握することができず、GPSデータなどの移動データから推測するしかない。そのGPSデータも自然環境下では測位精度が低く、野生動物装着型デバイスのバッテリー持続期間の問題からも、測位の間隔やタイミングが整ったデータを収集することは困難である。多種多様で、大量かつ欠落部分を含む移動データを分析する上で、これらの問題点を考慮した定量的な類型化手法は非常に重要となる。

2024年度は、2023年度から引き続き北海道の広い放牧地で自由に移動し活動する乳牛のGPSデータ等の収集を行い、6か月に渡る継続的な乳牛のGPSデータを収集することができた。また2023年度に収集した同様の乳牛のGPSデータと合わせて、これらを用いて類型化手法の検討を進めてきた。観光地などでの人の移動行動を対象とした類型化の既存手法では、対象エリア内を機能や空間的なつながりにもとづいて分割し、類型化に用いる。しかしこの手法は、隔たりのない自然環境を自由に移動する動物の移動行動には適用できないと考えられた。そこで本研究では実際の乳牛たちの移動行動の特徴にもとづいて、類型化の単位(ここでは1日24時間)ごとに対象エリアの分割を行えるようにした。また、1日毎の類型化結果をもとに、分析期間(約4カ月間)を通した類型化を行った。これにより、乳牛たちの日々の移動行動の類似性と、分析期間を通したその類似性の変化をもとに、移動行動の類型化を行うことが可能であることが示された。以上については、動物行動関係学会と地理情報科学関係学会の2カ所で報告を行った。\cite{kawase01, kawase02}また、2024年度に収集したGPSデータも合わせて対象とし、手法の妥当性の検討を進めている。

さらに、遅延耐性ネットワーク (DTN)技術を用いたワイヤレスセンサーネットワークのシミュレーションを行い、被覆面積の効率的な最大化手法について検討している。DTN技術を用いた野生動物装着型ワイヤレスセンサーネットワークにおいては、異なる群れの間を行き来したり、積極的に他の個体と接触したりする個体の存在が重要となる。そのような個体が、広くデータを伝播させる役割を担うことができると考えられるからである。類型化手法においては、「一緒に行動する群れ」を特定するだけでなく、「群れの間を行き来するような特徴的な少数派」を効率的に見つけ出すことを目的のひとつとしている。これらに着目し、継続的に研究を行っていく。