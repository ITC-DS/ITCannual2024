\subsection{データセンタハードウェアへのソフトウェア脆弱試験の適応(空閑 洋平)}

現在のデータセンタ環境では、AIモデル作成等を高速化するためにGPU等専用アクセラレータが広く使用されている。
専用アクセラレータを用いた計算環境は、既存のCPUを中心に構成されていたクラウド型の仮想マシンクラスタから、CPUをバイパスしてデバイス間で直接データ通信するヘテロジニアス構成に移行したことで、システム全体のブラックボックス化が進んでいる。
今後、専用アクセラレータを中心としたデータセンタ環境では、CPUをバイパスするデバイス間通信が増加することで、セキュリティ監視や脆弱性試験、管理手法、データ通信内容の可視化手法といった、クラウド型クラスタ環境で実施している運用課題が顕在化すると考えられる。
本年度は、昨年度に引き続き、PIM (Processing in Memory)型のデバイスメモリの設計開発を実施し、PoCを用いたLinux NVMeドライバの既存の脆弱性の再現手法を開発し、情報処理学会コンピュータシステム・シンポジウム2024で報告した\cite{ykuga4301xyyyy}。

その他の成果としては、NIIが主催するLLM勉強会の活動に参加し、172Bモデル作成に関する活動内容を報告した\cite{ykuga4301yyyyy}。
また、クラスタワークショップ in すずかけ台2024に参加し、mdxのRDMAネットワークに関する招待講演を実施した\cite{ykuga458xxxxx}。
