\subsection{データセンタハードウェアへのソフトウェア脆弱試験の適応(空閑 洋平)}

現在のデータセンタ環境では、LLMの事前学習やインストラクションチューニングを高速化するため、GPUなど専用アクセラレータが広く使用されている。
専用アクセラレータを用いた計算環境は、既存のCPUを中心に構成されていたソフトウェア環境に比べて、プロセッサやデバイスドライバ、デバイス間通信が専用に設計され、CPUをバイパスしてデバイス間で直接データ通信されるため、システムの機能拡張が困難になり、システム全体のブラックボックス化が進んでいる。
今後、専用アクセラレータを中心としたデータセンタ環境では、CPUをバイパスするデバイス間通信が増加することで、セキュリティ監視や脆弱性試験、管理手法、データ通信内容の可視化手法といった、普段CPU環境で実施している運用課題が顕在化すると考えられる。
本年度は、昨年度に引き続き、PIM (Processing in Memory)型のデバイスの設計開発を実施し、既存手法では困難であった、NIC, NVMe, GPU間データ通信の機能拡張手法を検討した。本研究は、情報処理学会のOS研究会から今年度山下記念賞を受賞し、情報処理学会の全国大会で講演を1件実施した\cite{ykuga45871732, ykuga43010880}。

また、昨年度から継続している東京大学のZoomデータを用いて実施した広域ネットワーク品質解析の手法に関する研究については、論文誌を1編報告し、情報処理学会IOT研究会から藤村記念ベストプラクティス賞、山下記念賞をそれぞれ受賞した\cite{ykuga45871761, ykuga43010877, ykuga43010874}。

その他の成果としては、NIIが主催するLLM勉強会の活動に参加し、活動内容を自然言語処理学会の招待論文で報告した\cite{ykuga45871752}。また、クラウドやHPC間のデータ転送を高速化するソフトウェアmscpに関する手法提案を国際会議ACM PEARCで報告し、Data Mover Challenge 2023でMost Innovative for HPC Users Awardを受賞した\cite{ykuga43404131, ykuga9999}。
