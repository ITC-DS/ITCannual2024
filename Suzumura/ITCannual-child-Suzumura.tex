% %半ページから1ページが文量

\subsection{推薦システムと脳波基盤モデルの研究(鈴村 豊太郎)}

%%%% 以下、2024年度

本節では2024年度の鈴村豊太郎の研究活動について報告する。鈴村は、ニュース、Eコマースおよびオンラインサービスにおける推薦システムの研究と、神経科学における基盤モデルの研究を進めている。

ニュース領域における推薦モデルでは、記事の新しさや多様性を維持しつつユーザの関心を的確に捉える必要がある。しかし、従来手法はクリック履歴や人気度に偏りがちで、ユーザの意図的な閲覧回避行動や記事の新しさを反映できないという技術的課題がある。そこで、ユーザの回避行動と表示頻度を同時に扱う Avoidance-Aware Recommender System(AWRS)\cite{awrs}を提案し、関連性と多様性、タイムリーさを高めた推薦を実現した。AWRSの研究成果は国際会議SIAM International Conference on Data Mining (SDM25)で発表された。また、ニュース推薦における人気バイアスに対処した。人気記事の影響をネガティブなフィードバックとしてモデル化する POPKを提案し、バイアスのないユーザー嗜好表現を学習することで、推薦精度と多様性の双方を改善した。引き続き、ニュース領域では日本経済新聞社との共同研究を進めている。

Eコマース分野では、知識グラフを活用したユーザターゲティングの研究と、組合せ最適化問題を高速で解くアニーリングマシン(AM)とGNNの併用に関する研究を行った。
多様なユーザの嗜好を適切に反映するには知識グラフによる豊富な属性情報の活用が有用であることから、知識グラフの埋め込みとGNNを組み合わせて新規ターゲティング候補を抽出するAudienceLinkNet \cite{aln}を提案した。これにより、スケーラビリティと精度を両立した最適化およびユーザへのターゲティングモデルの実用性を示した。これらの研究成果はSIGIR’24 にて発表後、実運用環境での検証を進めている。
また、商品の推薦や配信先の選定が大規模な組合せ最適化問題として定式化される一方、既存の最適化ソルバではスケールやリアルタイム性に限界があるという課題がある。本研究では、AMの部分解をGNNにフィードバックして大規模グラフ上で高品質な最適解を高速に導出する手法 \cite{annealing-gnn}を提案し、AMの精度とGNNのスケーラビリティを両立する基盤を示した。

オンラインサービスにおける推薦システムでは、ユーザ行動履歴だけでなく、テキストレビューや商品説明の意味情報を統合できれば強力なパーソナライズが実現可能である。しかし、LLMの生成力を推薦モデルに直接活かす際には、数値やテキスト、グラフなど異なるデータ構造を一貫して扱うモデリング手法が必要である。本研究では、Llama-2によるIn-Context Learningで生成したアイテムプロファイルをBERTで意味的埋め込みに変換し、GNNと統合するPrompting-Based Representation Learning(P4R)\cite{p4r}を提案した。テキストに含まれる豊富なコンテキストを継続的に活用しながらユーザとアイテムの関係性を学習する全く新しい推薦フレームワークを構築し、ROEGEN@RecSys’24 ワークショップで発表後、リアルタイム推論とモデル軽量化に向けた最適化を進めている。

脳波基盤モデルの研究として、脳活動を計測するEEG信号の基盤モデルにEEGセンサ(電極)の位置関係を捉えるGNNモデルを適用した研究を行った。EEG信号は高い時間分解能を有することから時系列基盤モデルが数多く提案されてきたが、異なる電極間の空間的関係を十分に活用できていない。本研究では、電極配置による信号伝搬の関係性をグラフ構造として表現し、EEGの基盤モデルBENDRにGNNを統合することで、時間的文脈を保ちつつ電極間の相互作用を学習可能な EEG-GraphAdapter \cite{ega} と、マスクドオートエンコーダとGNNを融合したGraph-Enhanced EEG Foundation Model \cite{gefm} を提案した。これにより、EEGデータを扱う下流タスクにおいてファインチューニングの計算コストを減らしつつ、高い予測精度を達成した。これらの研究成果はAAAI-25ワークショップでポスター発表され、\cite{gefm}は国際会議 IEEE Engineering in Medicine and Biology Society (EMBC’25)にも採択された。


%%%% 以下、2023年度

% 本節では2023年度の鈴村豊太郎の研究活動について報告する。鈴村は、グラフニューラルネットワーク(GNN)、大規模言語モデル(LLM)などの深層学習を基盤にした推薦システムの研究を進めている。

% 推薦システムは、ユーザの好みやアイテム行動パターンなどをモデル化し、どのようなアイテムを購入するかを予測する問題であり、現実世界のあらゆるサービスに推薦システムが用いられていると言っても過言ではない。本年度は、地理空間領域、ニュース領域、求人マッチング領域における推薦システム、およびLLMベースの推薦システムの研究を進めている。

%  時空間における推薦システムでは、自動車の走行軌跡データに基づき次に訪問する地点(Point-of-Interest、 POI)を予測するPOI推薦において、ユーザの空間的・時間的特徴をそれぞれグラフ構造として表現し、ユーザの行動パターンを捉えるトランスフォーマベースのモデルアーキテクチャ Mobility Graph Transformer (MobGT) を提案した。この研究成果は地理情報空間におけるトップカンファレンスである国際学会 ACM SIGSPATIAL'23 に採択され、発表を行った(\cite{mobgt})。これらはトヨタ自動車との共同研究の成果でもある。

%  ニュース領域における推薦モデルでは、全ユーザの記事閲覧行動データからなる Global News Graph と記事の内容から生成された Global Entity Graph を生成し、個別のユーザの閲覧履歴と組み合わせることで、ユーザの潜在的な記事閲覧パターンを捉える記事推薦システムモデル GLORY を提案した。この論文は国際学会 ACM RecSys 2023 に採択され、Best Full Paper Runner-Up Award 及び Best Student Paper Award を受賞した(\cite{glory})。
% また、新しい記事が常に追加されるニュースサイトにおいては、過去に掲載された記事がすぐに閲覧されなくなるため、これらの記事の鮮度がニュース記事の推薦において重要である。そこで、この記事の鮮度をその内容や人気度などかを考慮して予測し、ユーザにより適切な記事を推薦するモデルを提案する。ニュース領域では日本経済新聞社との共同研究を進めている。

%  求人マッチング領域では、ユーザの嗜好のみに基づく他の一般的な推薦システムとは異なり、求人側の意向も加味して、全体的な市場のマッチングを最大化する必要がある。そこで、マッチング数を報酬とした強化学習モデルを用いることで、end-to-end でマッチング数を最大化する手法を提案した。さらに、ユーザと求人に対する実際のマッチング数の少なさを補うために、ユーザの求人への応募、マッチングを二部グラフとして表現し、GNNモデル、グラフ拡張によって精度を向上させる手法を提案した。この研究成果は、それぞれ人工知能学会全国大会 (\cite{job-jsai})
% )及び AAAI Workshop (\cite{job-aaai}) に採択され発表を行い、現在ACMの情報抽出関連の国際会議 ACM SIGIR 2024 に投稿、査読中である。また、昨年度に引き続きエス・エム・エス社との共同研究を進めており、実サービスでの検証に向けシミュレーションによる評価だけでなく A/B テストも実施した。

%  大規模言語モデル LLM (Large Language Model) ベースの推薦システムでは、数値情報のエンコーティングに関する研究を行っている。推薦モデルでは、価格や数量など数値自体が意味を持つことが多いが、現状のLLMでは数字情報を理解することが苦手であり、推薦システムへの応用のネックとなっている。そこで、既存の推薦システム用の基盤モデルに対し、pre-training 時に加算など補助的な算術演算のタスクを行うことで、数値に関する表現を部分的に捉えることを示した。この研究成果は、AAAI Workshop にて採択、発表された (\cite{num-aaai})。

%  また、前述の地理空間推薦システムにおいて、POI の情報としてテキストに、画像などマルチモーダルな情報を活用した POI推薦フレームワークを設計し、プロトタイプを実装、性能評価を続けている。ユーザが訪問した地点のジャンル、紹介文、写真をそれぞれテキストとして表現し、LLMベースの時系列推薦モデルを用いてより高精度なPOI推薦を実現することも目指している。

% また、SIGIR 2023 \cite{sigir}、 WWW 2023 \cite{kg-kp}の論文に関しては2022年度年報に報告済みであり詳細を割愛するが、2023年度に発表を行ったのでこちらに触れておく。

% 学会活動としてはAAAI 2024の Organizing Chairs として Sponsor Chairを務めた。また、2023年度から人工知能学会における理事を務め、2024年5月開催の人工知能領域における国際シンポジウム isAI 2024 (The 16th JSAI International Symposia on AI)の委員長を務めて、学会の開催準備を進めている。



%%%% 以下、2022年度

% 本節では2022年度の鈴村豊太郎の研究活動について報告する。 鈴村は、 グラフ構造に対するニューラルネットワークを用いた表現学習 Graph Neural Network (以下、GNNと呼ぶ)の基礎研究及び応用研究に取り組んでいる。 グラフ構造は、 ノードと、 ノード同士を接続するエッジから構成されるデータ構造である。 インターネット上における社会ネットワーク、 購買行動、 サプライチェーン、 金融における決済データ、 交通ネットワーク、 蛋白質相互作用・神経活動・DNAシーケンス配列内の依存性、 物質の分子構造、 人間の骨格ネットワーク、 概念の関係性を表現した知識グラフなど、 グラフ構造として表現できる応用先は枚挙に暇がない。
% \par
% 当該研究領域における研究として、時系列・動的に変化する大規模グラフに対するGNNモデルの研究を行った。
% 動的グラフに対応するGNNモデルはすでに数多く提案されているが、いずれも短期的なデータの変化しか考慮されておらず、実世界で扱われている長期的なグラフデータでは長期的なコンテキストを捉えることができない問題が潜在的に存在していた。この問題に対して、時間幅が非常に長いグラフデータの性質も捉えることができる Spectral Waveletを提案した(AAAI 2023 \cite{aaai-deft}、 Transactions on Machine Learning Research (TLMR) \cite{feta})。また、知識グラフ上で足りない関係性を補完する手法を評価する方法として トポロジカルデータ解析(Topological Data Analysis) における Persistent Homologyの概念を用いて効率的に評価する手法を提唱した(WWW 2023 \cite{kg-kp})。
% % 当該研究領域における研究として、時系列・動的に変化する大規模グラフに対するGNNモデルの研究を行った。実世界では時間幅が非常に長いデータを扱うこともあるが既存の動的グラフへのGNNの研究ではそのような点を考慮していない。この問題に対して、学習可能な Spectral Waveletを提案し、AAAI 2023 \cite{aaai-deft}、 WWW 2023 \cite{deft}、 TLMR \cite{feta}に採択された。

% GNNに関する応用研究も進めている。金融領域においては不正検出に関するGNNモデルの検証を行い、取引ネットワークをヘテロジニアスなグラフ構造に拡張することによりモデル性能の向上を達成した(KDD'22 MLG Workshop \cite{eth-gnn})。マテリアルズ・インフォマティクスの分野においては、情報基盤センターの芝隼人先生とガラス物質の形成過程モデルに対して高精度なGNNモデルを提案し\cite{botan}、また当該分野における本質的な問題に対する手法として、インバランスなデータの問題を解消するための手法 \cite{xsig-limin}および外挿のためのモデル構築を行った\cite{xsig-takashige}。E-Commerceの領域においては知識グラフを用いた商品推薦手法を提案した (ACM SIGIR 2023 \cite{sigir})。

% また、理論モデルの実世界への検証と応用サイドから意味のある研究テーマを発掘するため、企業との共同研究とも進めている。まず、自動車の走行軌跡データから次の位置や経路を予測し、ロケーションリコメンデーションなどに応用するための手法をトヨタ自動車と探求している。走行軌跡データは緯度・経度及び時刻のシーケンスデータとなるが、それを用いると運転行動パターンを捉えることができる。
% % まずシーケンスデータからグラフ構造を構築し、そのグラフ構造からGraphormerというニューラルネットワークモデルを走行軌跡データのパターンを捉えられるようなニューラルネットワークのモデルを提案した。
% この行動パターンを捉えるために、シーケンスデータをグラフ構造として表現し、Graphormerをベースにした新たなモデルを提案し、他の既存手法よりも高い精度でパターンを予測することを確認した (ECML-PKDD \cite{stgtrans} 査読中)。
% % この新たなモデルを他の既存手法と比較し、より高い精度で走行パターンを予測する事を確認した。本研究の成果を ECML-PKDD \cite{stgtrans}に提出した。
% 来年度はモビリティにおける様々な領域に応用できるように、走行軌跡データや実世界の地図データなどから事前学習モデルを構築する予定である。また、その他に都市全体の二酸化炭素排出量を抑制するために交通流を分散するための手法をこれらの事前学習モデルと深層強化学習を用いて設計・実装する予定である。

% また、エス・エム・エス社との共同研究では、介護や医療領域における人材紹介の推薦システムに関する研究を行った。超高齢化社会に突入する中、介護や医療領域における人材不足は深刻であり、より精度の高い人材マッチングが不可欠である。この問題に対して、深層強化学習を用いた人材マッチング数の最適化手法を提案し、特定の求職者・事業者側に偏ってしまう従来の推薦・マッチング手法に対して、偏りを解消できることを確認した (人工知能学会\cite{sms} 6月発表予定)。来年度に関しては更に実データでの検証を進め、企業側での要望を取り入れ、実ビジネスが持つ制約条件を取り入れた最適化モデルを提案していく予定である。また、モデルにおいて求職者と求人側での動的な二部グラフの関係性及び知識グラフを用いてより精度高いモデルを構築していく予定である。

%  日本経済新聞社(以下、日経)との共同研究も2022年10月から開始している。日経ではニュースサイトにおける記事に対してより高度な機械学習による推薦システムを目指しており、2022年度は日経側での問題設定やデータの理解を図り、研究テーマの設定に主に取り組んだ。2023年度は推薦問題や広告配信など様々な領域に応用できるように、ユーザーの行動モデルを統一的に表現する事前学習モデルを構築するべく、自己教師付き学習 (Self-Supervised Learning)を用いた手法を設計・実装する予定である。

%  また、GNNの概要と最新研究動向に関する記事を人工知能学会誌\cite{jsai-gnn}に寄稿し、Federated Learning(連合学習)の英語書籍向けに Federated Learningを用いた金融不正検知に関する手法を執筆した\cite{fl-book}。mdxプロジェクトに関する第一弾の国際学会論文として IEEE CBDCom\cite{mdx}にて論文発表を行った。


% % How Expressive are Transformers in Spectral Domain for Graphs? \cite{feta}
% % Learnable Spectral Wavelets on Dynamic Graphs to Capture Global Interactions \cite{deft}
% % Can Persistent Homology provide an efficient alternative for Evaluation of Knowledge Graph Completion Methods? \cite{kg-kp}


% % Spatio-Temporal Meta-Graph Learning for Traffic Forecasting \cite{megacrn}

% % Ethereum Fraud Detection with Heterogeneous Graph Neural Networks \cite{eth-gnn}

% % Federated Learning for Collaborative Financial Crimes Detection \cite{fl-book}



% %  本節では2021 年度の鈴村豊太郎の研究活動について報告する。 2021年4月に本学に着任し、 グラフ構造に関するニューラルネットワークを用いた表現学習 Graph Neural Network (以下、GNNと呼ぶ)の基礎研究及びその様々な応用研究に取り組んでいる。 グラフ構造は、 ノードと、 ノード同士を接続するエッジから構成されるデータ構造である。 インターネット上における社会ネットワーク、 購買行動、 サプライチェーン、 金融における決済データ、 交通ネットワーク、 蛋白質相互作用・神経活動・DNAシーケンス配列内の依存性、 物質の分子構造、 人間の骨格ネットワーク、 概念の関係性を表現した知識グラフなど、 グラフ構造として表現できる応用先は枚挙に暇がない。
% % \par
% % 当該研究領域において、時系列・動的に変化する大規模グラフに対するGNNモデルの研究を行った。分散計算環境においてスケールするGNNモデルを提唱し、その成果は高性能計算分野におけるトップカンファレンスSC2021\cite{suzumura-sc2021}に採択された。 また、金融領域における不正検知手法として、TransformerアーキテクチャをベースにしたGNN手法を提案し、国際会議 IEEE SMDS 2021\cite{suzumura-smds21}に採択された。 また、 GNNに関する招待講演\cite{suzumura-canon2021}を行った。

% % これらの研究に続いて、推薦システムへのGNNモデルに関する研究を開始している。実データ・実問題に基づいた、社会実装を見据えた研究を進めるべく、医療・介護領域における人材推薦としてエス・エム・エス社、自動車における経路推薦としてトヨタ社と共同研究を2023年4月から本格的に開始する。また、国立研究開発法人物質・材料研究機構NIMSが主導する「マテリアル先端リサーチインフラ」プロジェクトの本学拠点の一貫で、材料情報科学 Materials Informaticsへの研究も開始している。
% %  データ科学・データ利活用のためのクラウド基盤 mdx プロジェクトにおいて、 今年度は 課金付き運用開始に向けたシステム拡張、スポットVM、データ共有機構(Platform-as-a-Service)に向けた設計を進めた。 また、 mdxに関する講演活動を国内外において行った\cite{suzumura-axies2021、suzumura-nanotec2021、 suzumura-nci2021}。 mdxの論文においては、国際会議IEEE IC2E2022(10th IEEE International Conference on Cloud Engineering) に2022年3月末に投稿した。 
% % %\cite{suzumura-mdx2022}においても論文を公開した。

