\subsection{ダイナミックビジョンシステムとその応用展開(末石 智大)}

高速画像処理および高速光学系制御を活用する、動的検査技術とヒューマンインターフェースに関するダイナミックビジョンシステムの研究を実施した。

実世界の動的かつ複雑な現象を適応的にデータ化し、意味のある形で活用する動的検査技術として,本年度もマイクロサッカードと呼ばれる眼球微振動を対象として実施した。頭部固定を必要としない計測条件でのマイクロサッカード検出は、被験者を拘束する負荷や時間効率の観点から高い有効性が期待される。リラックスした状態の人間のマイクロサッカード検出に向けて、回転ミラーや液体可変焦点レンズなどの光学素子を高速に制御しつつ高解像度合焦画像計測を達成することで、ダイナミックビジョンシステムにおける動的検査への基礎技術の研鑽を本年度も進めている。本技術の定量的評価のための動的眼球模型の両眼化など、周辺技術の開発による成果も実現している。

また、昨年度に引き続き球技スポーツであるテニスのライン判定を目的とした高速ビジョンと落下位置予測技術の融合システムの研究も進めており、本年度は国際学会においても受賞を獲得した。高速なカルマンフィルタに基づくテニスボールの落下予測位置に対して先行した光学系制御による、地面テクスチャの高解像度撮影を実現することで、比較的単純な差分処理によりボール着地痕跡を可視化する技術である。
ヒューマンインターフェースに関しては、ボールの高速落下位置予測を活用した、ダイナミックプロジェクションマッピングによる先読み情報提示や、円筒鏡面と即時フィードバックを活用した視点依存ディスプレイの一種であるダイナミックアナモルフォーシスシステムも実現している。特に後者のアナモルフォーシスシステムは、国内学会における優秀講演賞も獲得している。

本年度は総じて、運動予測も包含させた高速センシング技術を洗練し、検査やダイナミックプロジェクションマッピングへの実世界応用の展開を成熟させた。
