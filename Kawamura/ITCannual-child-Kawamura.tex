\subsection{第一原理計算とデータ科学・機械学習による物質科学研究(河村 光晶)}

物質を構成する電子や原子に対して、基本法則となる(相対論的)量子力学や統計物理学に基づく理論的研究と、様々な環境(温度、圧力等)やプローブ(電子線、可視光、X線、中性子線、物理量測定等)における実験的研究の両者を協調的に行う事で、既存の現象の理解や新たなる物質探索をより効果的に進めることができる。
実験との比較を行うにあたり、物性物理学の理論を多様な組成$\cdot$構造を持つ現実の物質に適用するためには計算機によるシミュレーションが不可欠となり、スパコンやmdxのような高速$\cdot$大規模な並列計算資源およびデータ格納環境が利用される。
我々はそのような大規模計算機における物質科学シミュレーションの手法やプログラムの開発、およびそれを実際の物質$\cdot$現象に適用する研究を行っている。
%利根さん
~\cite{kawamura_hanai_tone_continuousp}
%HPHi
~\cite{kawamura_hphi}
%WO$_2$
~\cite{kawamura_wo2}
%KMgH$_3$
~\cite{kawamura_kmgh3}
%o-MAX相
~\cite{kawamura_omax}
%Pd-deposited Cu(111) surface
~\cite{kawamura_pdcu111}